\documentclass[11pt]{article}
\pagestyle{empty}
\usepackage{color}
\usepackage{fancyhdr}
\usepackage{lastpage}
\usepackage{amsmath}
\pagestyle{fancy}
\renewcommand{\headrulewidth}{0pt}
\cfoot[R]{\thepage~of~\pageref{LastPage}}
\addtolength{\oddsidemargin}{-.875in}
\addtolength{\evensidemargin}{-.875in}
\addtolength{\textwidth}{1.75in}
\addtolength{\topmargin}{-.875in}
\addtolength{\textheight}{1.75in}

\begin{document}
\begin{center} {\Large\bf FA, Homework 7}  \\ Quentin McGaw (qm301) \\ 03/24/17
\end{center}

\begin{quote}
What you need is that your brain is open.  -- Paul Erd\H{o}s
\end{quote}

\begin{enumerate}
\item \textbf{\textcolor{blue}{Determine an LCS of $10010101$ and $010110110$ using the algorithm studied.}}
    \\ Solution
    
    
\item \textbf{\textcolor{blue}{Write all the parenthesizations of $ABCDE$.
Associate them in a natural way with (setting $n=5$) the
terms $P(i)P(n-i)$, $i=1,2,3,4$ given in the recursion for $P(n)$.}}
    \\ For $P(1)P(4)$, we have $4$ parenthesizations:
    \\ $A(B(C(DE)))$
    \\ $A(B((CD)E))$
    \\ $A((BC)(DE))$
    \\ $A((B(CD))E)$
    \\ $A(((BC)D)E)$
    \\ For $P(2)P(3)$, we have $2$ parenthesizations:
    \\ $(AB((CD)E)$
    \\ $(AB)(C(DE))$
    \\ For $P(3)P(2)$, we have $2$ parenthesizations:
    \\ $((AB)C)(DE)$
    \\ $(A(BC))(DE)$
    \\ For $P(4)P(1)$, we have $5$ parenthesizations:
    \\ $(A(B(CD)))E$
    \\ $(A((BC)D))E$
    \\ $((AB)(CD))E$
    \\ $(((AB)C)D)E$
    \\ $((A(BC))D)E$



    
    
    
\item \textbf{\textcolor{blue}{Let $x_1,\ldots,x_m$ be a sequence of distinct real numbers. For $1\leq i\leq m$ let $INC[i]$ denote the length of the longest
increasing subsequence ending with $x_i$.  Let $DEC[i]$ denote the 
length of the longest decreasing subsequence ending with $x_i$.  
{\tt Caution:} The subsequence must {\em use} $x_i$.  For example,
$20,30,4,50, 10$.  Now $INC[5]=2$ because of $4,10$ -- we do
{\em not} count $20,30,50$.}}
    \begin{enumerate}
    \item \textbf{\textcolor{blue}{Find an efficient method for finding the values $INC[i]$, $1\leq i\leq n$. (You should find $INC[i]$ based on the previously found $INC[j]$, $1\leq j< i$.  Your algorithm should take time $O(i)$ for each
    particular $i$ and thus $O(n^2)$ overall.)}}
        \\ Solution
    \item \textbf{\textcolor{blue}{Let $LIS$ denote the length of the longest increasing subsequence of $x_1,\ldots,x_m$.  Show how to find $LIS$ from
    the values $INC[i]$. Your algorithm, {\em starting with} the $INC[i]$,
    should take time $O(n)$. Similarly, let $DIS$ denote the length of the longest
    decreasing subsequence of $x_1,\ldots,x_m$.  Show how to find $DIS$ from
    the values $DEC[i]$. }}
        \\ Solution
    \item \textbf{\textcolor{blue}{Suppose $i<j$.  {\em Prove} that it is impossible to have $INC[i]=INC[j]$ {\em and} $DEC[i]=DEC[j]$.  ({\tt Hint:} Show that {\em if} $x_i < x_j$ then $INC[j] \geq INC[i] + 1$.)}}
        \\ Solution
    \item \textbf{\textcolor{blue}{Deduce the following celebrated result (called the Monotone Subsequence Theorem) of Paul Erd\H{o}s and George Szekeres:  Let $m=ab+1$.  Then any sequence $x_1,\ldots,x_m$ of distinct real numbers either $LIS>a$ or $DIS>b$.  (Idea: Assume not and look at the pairs $(INC[i],DEC[i])$.)  Paul Erd\H{o}s was a great twentieth century mathematician,  whose work remains highly influential in many areas.}}
        \\ Solution
    \end{enumerate}
    
    
\item \textbf{\textcolor{blue}{Find an optimal parenthesization of a matrix-chain product whose sequence of dimensions is $5,10,3,12,5,50,6$.}}
    
\item \textbf{\textcolor{blue}{Some exercises in logarithms:}}
    \begin{enumerate}
    \item \textbf{\textcolor{blue}{Write $\lg(4^n/\sqrt{n})$ in simplest form.  What is its asymptotic value.}}
        \\ $\lg(4^n/\sqrt{n}) = \lg(4^n) - \lg(n^{1/2}) = \lg(4)n - \frac{1}{2}\lg(n)$ so its asymptotic value is infinity as $n$ dominates $\lg(n)$.
    \item \textbf{\textcolor{blue}{Which is bigger, $5^{313340}$ or $7^{271251}$? Give reason. (You can use a calculator but you can't use any numbers bigger than $10^9$.)}}
        \\ Suppose $5^{313340}$ is lower than $7^{271251}$.
        \\ $5^{313340} < 7^{271251} \Rightarrow 313340\lg(5) < 271251\lg(7) \Rightarrow 219015.26 < 229233.68$
        \\ Our assumption was correct, so $7^{271251}$ is bigger.
    \item \textbf{\textcolor{blue}{Simplify $n^2\lg(n^2)$ and $\lg^2(n^3)$.}}
        \\ $n^2\lg(n^2) = 2n^2\lg(n)$
        \\ $\lg^2(n^3) = (3\lg(n))(3\lg(n)) = 9\lg^2(n)$
    \item \textbf{\textcolor{blue}{Solve (for $x$) the equation $e^{-x^2/2}=\frac{1}{n}$.}}
        \\ $e^{-x^2/2}=\frac{1}{n} \Rightarrow 1=\frac{e^{x^2/2}}{n} \Rightarrow \ln(1)=\ln(\frac{e^{x^2/2}}{n}) \Rightarrow 0 = \ln(e^{x^2/2}) - \ln(n) \Rightarrow 0 = \frac{x^2}{2} - \ln(n) \Rightarrow 2\ln(n) = x^2 \Rightarrow x = \mp \sqrt{2\ln(n)}$
    \item \textbf{\textcolor{blue}{Write $\log_n2^n$ and $\log_nn^2$ in simple form.}}
        \\ $\log_n2^n = \frac{\log(2^n)}{\log(n)} = \frac{n\log(2)}{\log(n)}$
        \\ $\log_nn^2 = \frac{\log(n^2)}{\log(n)} = \frac{2\log(n)}{\log(n)} = 2$
    \item \textbf{\textcolor{blue}{What is the relationship between $\lg n$ and $\log_3n$?}}
        \\ $\lg n = \frac{\log(n)}{\log(10)}$ and $\log_3n = \frac{\log(n)}{\log(3)}$.
        \\\\ So the base of $\lg n$ is 10 whereas the base of $\log_3n$ is 3.
    \item \textbf{\textcolor{blue}{Assume $i< n$.  How many times need $i$ be doubled before it reaches (or exceeds) $n$?}}
        \\ $\log_2(\frac{n}{i})$
    \item \textbf{\textcolor{blue}{Write $\lg[n^ne^{-n}\sqrt{2\pi n}]$ precisely as a sum in simplest form.  What is it asymptotic to as $n\ra \infty$?  What is interesting about the bracketed expression?}}
        \\ $\lg[n^ne^{-n}\sqrt{2\pi n}] = \lg(n^n) + \lg(e^{-n}) + \lg(\sqrt{2\pi n}) = n\lg(n) - n\lg(e) + 0.5\lg(2\pi n)$ so the asymptotic as $n$ goes to $\infty$ is $\infty$.
    \end{enumerate}
\end{enumerate}

\begin{quote}
There is a theory which states that if ever anybody discovers exactly what
the Universe is for and why it is here, it will instantly disappear and be
replaced by something even more bizarre and inexplicable. There is another
theory which states that this has already happened.
\\ Douglas Adams
\end{quote}

\end{document}

