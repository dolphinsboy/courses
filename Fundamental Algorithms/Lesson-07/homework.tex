\documentclass[11pt]{article}
\pagestyle{empty}
\usepackage{color}
\usepackage{fancyhdr}
\usepackage{lastpage}
\usepackage{amsmath}
\usepackage{minted}
\usepackage[options]{algorithm2e}
\pagestyle{fancy}
\renewcommand{\headrulewidth}{0pt}
\cfoot[R]{\thepage~of~\pageref{LastPage}}
\addtolength{\oddsidemargin}{-.875in}
\addtolength{\evensidemargin}{-.875in}
\addtolength{\textwidth}{1.75in}
\addtolength{\topmargin}{-.875in}
\addtolength{\textheight}{1.75in}

\begin{document}
\begin{center} {\Large\bf FA, Homework 7}  \\ Quentin McGaw (qm301) \\ 03/24/17
\end{center}

\begin{quote}
What you need is that your brain is open.  -- Paul Erd\H{o}s
\end{quote}

\begin{enumerate}
\item \textbf{\textcolor{blue}{Determine an LCS of $10010101$ and $010110110$ using the algorithm studied.}}
    \begin{algorithm}
        \SetKwFunction{lcslength}{LCS-LENGTH}
        \Indm\lcslength{x, y} \\
        \Indp
            m = x.length \\
            n = y.length \\
            Let b[1..m, 1..n] and c[0..m, 0..n] be new matrices. \\
            \tcc{We now set the first row and first column of c to 0.} \\
            \For{i = 1 to m}{
                c[i, 0] = 0 \\
            }
            \For{i = 0 to n}{
                c[0, i] = 0 \\
            }
            \For{i = 1 to m}{
                \For{j = 1 to n}{
                    \uIf{$x_i$ == $y_j$}{
                        c[i, j] = c[i - 1, j - 1] + 1 \tcp*{Cell = top-left cell, and point to it} \\
                        b[i, j] = \nwarrow\hspace{-3pt} \\
                    }
                    \uElseIf{c[i - 1, j] $\geq$ c[i, j - 1]}{
                        \tcp{top $\geq$ left} \\
                        c[i, j] = c[i - 1, j] \tcp*{Cell = top cell, and point to it} \\
                        b[i, j] = \uparrow\hspace{-3pt} \\
                    }
                    \Else{
                        c[i, j] = c[i, j - 1] \tcp*{Cell = left cell, and point to it} \\
                        b[i, j] = \leftarrow\hspace{-3pt} \\
                    }
                }
            }
            \Return{c, b}
        \caption{LCS-LENGTH, where x and y are two sequences}
    \end{algorithm}
    \\ Following this algorithm we have the following table initially:
    \\
    \begin{tabular}{ c | c c c c c c c c c c } 
        - & - & 0 & 1 & 0 & 1 & 1 & 0 & 1 & 1 & 0 \\
        \hline
        - & 0 & 0 & 0 & 0 & 0 & 0 & 0 & 0 & 0 & 0 \\
        1 & 0 & x & x & x & x & x & x & x & x & x \\
        0 & 0 & x & x & x & x & x & x & x & x & x \\
        0 & 0 & x & x & x & x & x & x & x & x & x \\
        1 & 0 & x & x & x & x & x & x & x & x & x \\
        0 & 0 & x & x & x & x & x & x & x & x & x \\
        1 & 0 & x & x & x & x & x & x & x & x & x \\
        0 & 0 & x & x & x & x & x & x & x & x & x \\
        1 & 0 & x & x & x & x & x & x & x & x & x \\
    \end{tabular}
    $\rightarrow$
    \begin{tabular}{ c | c c c c c c c c c c } 
        - & - & 0 & 1 & 0 & 1 & 1 & 0 & 1 & 1 & 0 \\
        \hline
        - & 0 & 0 & 0 & 0 & 0 & 0 & 0 & 0 & 0 & 0 \\
        1 & 0 & 0 $\leftarrow$ & x & x & x & x & x & x & x & x \\
        0 & 0 & x & x & x & x & x & x & x & x & x \\
        0 & 0 & x & x & x & x & x & x & x & x & x \\
        1 & 0 & x & x & x & x & x & x & x & x & x \\
        0 & 0 & x & x & x & x & x & x & x & x & x \\
        1 & 0 & x & x & x & x & x & x & x & x & x \\
        0 & 0 & x & x & x & x & x & x & x & x & x \\
        1 & 0 & x & x & x & x & x & x & x & x & x \\
    \end{tabular}
    $\rightarrow$ \\
    \begin{tabular}{ c | c c c c c c c c c c } 
        - & - & 0 & 1 & 0 & 1 & 1 & 0 & 1 & 1 & 0 \\
        \hline
        - & 0 & 0 & 0 & 0 & 0 & 0 & 0 & 0 & 0 & 0 \\
        1 & 0 & 0 $\leftarrow$ & 1 $\nwarrow$ & x & x & x & x & x & x & x \\
        0 & 0 & x & x & x & x & x & x & x & x & x \\
        0 & 0 & x & x & x & x & x & x & x & x & x \\
        1 & 0 & x & x & x & x & x & x & x & x & x \\
        0 & 0 & x & x & x & x & x & x & x & x & x \\
        1 & 0 & x & x & x & x & x & x & x & x & x \\
        0 & 0 & x & x & x & x & x & x & x & x & x \\
        1 & 0 & x & x & x & x & x & x & x & x & x \\
    \end{tabular}
    \\ We then continue on that same first row and we obtain: \\
    \begin{tabular}{ c | c c c c c c c c c c } 
        - & - & 0 & 1 & 0 & 1 & 1 & 0 & 1 & 1 & 0 \\
        \hline
        - & 0 & 0 & 0 & 0 & 0 & 0 & 0 & 0 & 0 & 0 \\
        1 & 0 & 0 $\leftarrow$ & 1 $\nwarrow$ & 1 $\leftarrow$ & 1 $\leftarrow$ & 1 $\leftarrow$ & 1 $\leftarrow$ & 1 $\leftarrow$ & 1 $\leftarrow$ & 1 $\leftarrow$ \\
        0 & 0 & x & x & x & x & x & x & x & x & x \\
        0 & 0 & x & x & x & x & x & x & x & x & x \\
        1 & 0 & x & x & x & x & x & x & x & x & x \\
        0 & 0 & x & x & x & x & x & x & x & x & x \\
        1 & 0 & x & x & x & x & x & x & x & x & x \\
        0 & 0 & x & x & x & x & x & x & x & x & x \\
        1 & 0 & x & x & x & x & x & x & x & x & x \\
    \end{tabular}
    \\ We then do the same for the next rows and finally obtain: \\
    \begin{tabular}{ c | c c c c c c c c c c } 
        - & - & 0 & 1 & 0 & 1 & 1 & 0 & 1 & 1 & 0 \\
        \hline
        - & 0 & 0 & 0 & 0 & 0 & 0 & 0 & 0 & 0 & 0 \\
        1 & 0 & 0 $\leftarrow$ & 1 $\nwarrow$ & 1 $\leftarrow$ & 1 $\nwarrow$ & 1 $\nwarrow$ & 1 $\leftarrow$ & 1 $\nwarrow$ & 1 $\nwarrow$ & 1 $\leftarrow$ \\
        0 & 0 & 1 $\nwarrow$ & 1 $\uparrow$ & 2 $\nwarrow$ & 2 $\leftarrow$ & 2 $\leftarrow$ & 2 $\nwarrow$ & 2 $\leftarrow$ & 2 $\leftarrow$ & 2 $\nwarrow$ \\
        0 & 0 & 1 $\nwarrow$ & 1 $\uparrow$ & 2 $\nwarrow$ & 2 $\uparrow$ & 2 $\uparrow$ & 3 $\nwarrow$ & 3 $\leftarrow$ & 3 $\leftarrow$ & 3 $\nwarrow$ \\
        1 & 0 & 1 $\uparrow$ & 2 $\nwarrow$ & 2 $\uparrow$ & 3 $\nwarrow$ & 3 $\nwarrow$ & 3 $\uparrow$ & 4 $\nwarrow$ & 4 $\nwarrow$ & 4 $\leftarrow$ \\
        0 & 0 & 1 $\nwarrow$ & 2 $\uparrow$ & 3 $\nwarrow$ & 3 $\uparrow$ & 3 $\uparrow$ & 4 $\nwarrow$ & 4 $\uparrow$ & 4 $\uparrow$ & 5 $\nwarrow$ \\
        1 & 0 & 1 $\uparrow$ & 2 $\nwarrow$ & 3 $\uparrow$ & 4 $\nwarrow$ & 4 $\nwarrow$ & 4 $\uparrow$ & 5 $\nwarrow$ & 5 $\nwarrow$ & 5 $\uparrow$ \\
        0 & 0 & 1 $\nwarrow$ & 2 $\uparrow$ & 3 $\nwarrow$ & 4 $\uparrow$ & 4 $\uparrow$ & 5 $\nwarrow$ & 5 $\uparrow$ & 5 $\uparrow$ & 6 $\nwarrow$ \\
        1 & 0 & 1 $\uparrow$ & 2 $\nwarrow$ & 3 $\uparrow$ & 4 $\nwarrow$ & 5 $\nwarrow$ & 5 $\uparrow$ & 6 $\nwarrow$ & 6 $\nwarrow$ & 6 $\uparrow$ \\
    \end{tabular}
    \\\\ To go faster:    
    \begin{itemize}
        \item Each cell where $x_i$ = $y_i$ has $\nwarrow$. Put it everywhere first.
        \item When the time comes, $\nwarrow$ cells got the value \textbf{left-up + 1}.
        \item Each remaining cell is either top (\textbf{priority}) if it is bigger or left otherwise (\textbf{from left to right})
    \end{itemize}
    \\ Starting from the bottom right of the table, each time we encounter $\nwarrow$, $x_i = y_i$ is the value of the nth position where n is the value in the cell. In this case, we obtain the following:
    \begin{itemize}
        \item Position 6: 0
        \item Position 5: 1
        \item Position 4: 1
        \item Position 3: 0
        \item Position 2: 0
        \item Position 1: 1
    \end{itemize}
    So one LCS is 100110 \\
    \begin{tabular}{ c | c c c c c c c c c c } 
        - & - & 0 & \textcolor{red}{1} & \textcolor{red}{0} & 1 & 1 & \textcolor{red}{0} & \textcolor{red}{1} & \textcolor{red}{1} & \textcolor{red}{0} \\
        \hline
        - & 0 & 0 & 0 & 0 & 0 & 0 & 0 & 0 & 0 & 0 \\
        1 & 0 & 0 $\leftarrow$ & \textcolor{blue}{\textbf{1} $\nwarrow$} & 1 $\leftarrow$ & 1 $\nwarrow$ & 1 $\nwarrow$ & 1 $\leftarrow$ & 1 $\nwarrow$ & 1 $\nwarrow$ & 1 $\leftarrow$ \\
        0 & 0 & 1 $\nwarrow$ & 1 $\uparrow$ & \textcolor{blue}{\textbf{2} $\nwarrow$} & \textbf{2} $\leftarrow$ & \textbf{2} $\leftarrow$ & 2 $\nwarrow$ & 2 $\leftarrow$ & 2 $\leftarrow$ & 2 $\nwarrow$ \\
        0 & 0 & 1 $\nwarrow$ & 1 $\uparrow$ & 2 $\nwarrow$ & 2 $\uparrow$ & 2 $\uparrow$ & \textcolor{blue}{\textbf{3}} $\nwarrow$ & 3 $\leftarrow$ & 3 $\leftarrow$ & 3 $\nwarrow$ \\
        1 & 0 & 1 $\uparrow$ & 2 $\nwarrow$ & 2 $\uparrow$ & 3 $\nwarrow$ & 3 $\nwarrow$ & 3 $\uparrow$ & \textcolor{blue}{\textbf{4} $\nwarrow$} & 4 $\nwarrow$ & 4 $\leftarrow$ \\
        0 & 0 & 1 $\nwarrow$ & 2 $\uparrow$ & 3 $\nwarrow$ & 3 $\uparrow$ & 3 $\uparrow$ & 4 $\nwarrow$ & \textbf{4} $\uparrow$ & 4 $\uparrow$ & 5 $\nwarrow$ \\
        1 & 0 & 1 $\uparrow$ & 2 $\nwarrow$ & 3 $\uparrow$ & 4 $\nwarrow$ & 4 $\nwarrow$ & 4 $\uparrow$ & 5 $\nwarrow$ & \textcolor{blue}{\textbf{5} $\nwarrow$} & 5 $\uparrow$ \\
        0 & 0 & 1 $\nwarrow$ & 2 $\uparrow$ & 3 $\nwarrow$ & 4 $\uparrow$ & 4 $\uparrow$ & 5 $\nwarrow$ & 5 $\uparrow$ & 5 $\uparrow$ & \textcolor{blue}{\textbf{6} $\nwarrow$} \\
        1 & 0 & 1 $\uparrow$ & 2 $\nwarrow$ & 3 $\uparrow$ & 4 $\nwarrow$ & 5 $\nwarrow$ & 5 $\uparrow$ & 6 $\nwarrow$ & 6 $\nwarrow$ & \textbf{6} $\uparrow$ \\
    \end{tabular}
    
    \\\\ We can also do this table the other way around, flipping "x" and "y": \\
    \begin{tabular}{ c | c c c c c c c c c } 
        - & - & 1 & 0 & \textcolor{red}{0} & \textcolor{red}{1} & \textcolor{red}{0} & \textcolor{red}{1} & \textcolor{red}{0} & \textcolor{red}{1} \\
        \hline
        - & 0 & 0 & 0 & 0 & 0 & 0 & 0 & 0 & 0 \\
        0 & 0 & 0 $\uparrow$ & 1 $\nwarrow$ & \textbf{1} $\nwarrow$ & 1 $\leftarrow$ & 1 $\nwarrow$ & 1 $\leftarrow$ & 1 $\nwarrow$ & 1 $\leftarrow$ \\
        1 & 0 & 1 $\nwarrow$ & 1 $\uparrow$ & 1 $\uparrow$ & \textbf{2} $\nwarrow$ & 2 $\leftarrow$ & 2 $\nwarrow$ & 2 $\leftarrow$ & 2 $\nwarrow$ \\
        0 & 0 & 1 $\uparrow$ & 2 $\nwarrow$ & 2 $\nwarrow$ & 2 $\uparrow$ & \textbf{3} $\nwarrow$ & 3 $\leftarrow$ & 3 $\nwarrow$ & 3 $\leftarrow$ \\
        1 & 0 & 1 $\nwarrow$ & 2 $\uparrow$ & 2 $\uparrow$ & 3 $\nwarrow$ & \textbf{3} $\uparrow$ & 4 $\nwarrow$ & 4 $\leftarrow$ & 4 $\nwarrow$ \\
        1 & 0 & 1 $\nwarrow$ & 2 $\uparrow$ & 2 $\uparrow$ & 3 $\nwarrow$ & 3 $\uparrow$ & \textbf{4} $\nwarrow$ & 4 $\uparrow$ & 5 $\nwarrow$ \\
        0 & 0 & 1 $\uparrow$ & 2 $\nwarrow$ & 3 $\nwarrow$ & 3 $\uparrow$ & 4 $\nwarrow$ & 4 $\uparrow$ & \textbf{5} $\nwarrow$ & 5 $\uparrow$ \\
        1 & 0 & 1 $\nwarrow$ & 2 $\uparrow$ & 3 $\uparrow$ & 4 $\nwarrow$ & 4 $\uparrow$ & 5 $\nwarrow$ & \textbf{5} $\uparrow$ & 6 $\nwarrow$ \\
        1 & 0 & 1 $\nwarrow$ & 2 $\uparrow$ & 3 $\uparrow$ & 4 $\nwarrow$ & 4 $\uparrow$ & 5 $\nwarrow$ & 5 $\uparrow$ & \textbf{6} $\nwarrow$ \\
        0 & 0 & 1 $\uparrow$ & 2 $\nwarrow$ & 3 $\nwarrow$ & 4 $\uparrow$ & 5 $\nwarrow$ & 5 $\uparrow$ & 6 $\nwarrow$ & \textbf{6} $\uparrow$ \\
    \end{tabular}
    \\ And we obtain the LCS 010101. 
    \\ Note that this is \textbf{equivalent} to switch around $\uparrow$ and $\leftarrow$ in the first table !!
    \\ Another way is just to go up all the time but when $x_i$ = $y_i$ then go up-left ! So you just have to find out c, not b really.
    \\ Note that I designed the following Python code to generate the table:
    \begin{minted}{python}
def print_matrix(m):
    for i in range(len(m)):
        for j in range(len(m[i])):
            print str(m[i][j]) + " ",
        print
    print
    
    
def lcs_length(x, y):
    m = len(x) + 1 # +1 for zeroes
    n = len(y) + 1
    b = [["." for _ in range(n)] for _ in range(m)]
    c = [[None for _ in range(n)] for _ in range(m)]
    for i in range(1, m):
        c[i][0] = 0
    for i in range(n):
        c[0][i] = 0
    for i in range(1,m):
        for j in range(1,n):
            if x[i-1] == y[j-1]:
                c[i][j] = c[i-1][j-1] + 1
                b[i][j] = '\\'
            elif c[i-1][j] >= c[i][j-1]:
                c[i][j] = c[i-1][j]
                b[i][j] = "|"
            else:
                c[i][j] = c[i][j-1]
                b[i][j] = "<"
    return c, b

def put_together(c, b, x, y):
    rows = len(x) + 2 # +1 for zeroes, +1 for x
    cols = len(y) + 2
    matrix = [[None for _ in range(cols)] for _ in range(rows)]
    matrix[0][0] = "-"
    matrix[1][0] = "-"
    matrix[0][1] = "-  "
    for i in range(2, rows):
        matrix[i][0] = x[i-2]
    for i in range(2, cols):
        matrix[0][i] = str(y[i-2]) + "  "
    for i in range(1, rows):
        for j in range(1, cols):
            matrix[i][j] = str(c[i-1][j-1]) + " " + b[i-1][j-1]
    return matrix

if __name__ == '__main__':
    y = "10010101"
    x = "010110110"
    c, b = lcs_length(x, y)
    m = put_together(c, b, x, y)
    print_matrix(c)
    print_matrix(b)
    print_matrix(m)
    \end{minted}
    \\ Which would print:
    \begin{minted}{python}
-  -    1    0    0    1    0    1    0    1   
-  0 .  0 .  0 .  0 .  0 .  0 .  0 .  0 .  0 . 
0  0 .  0 |  1 \  1 \  1 <  1 \  1 <  1 \  1 < 
1  0 .  1 \  1 |  1 |  2 \  2 <  2 \  2 <  2 \ 
0  0 .  1 |  2 \  2 \  2 |  3 \  3 <  3 \  3 < 
1  0 .  1 \  2 |  2 |  3 \  3 |  4 \  4 <  4 \ 
1  0 .  1 \  2 |  2 |  3 \  3 |  4 \  4 |  5 \ 
0  0 .  1 |  2 \  3 \  3 |  4 \  4 |  5 \  5 | 
1  0 .  1 \  2 |  3 |  4 \  4 |  5 \  5 |  6 \ 
1  0 .  1 \  2 |  3 |  4 \  4 |  5 \  5 |  6 \ 
0  0 .  1 |  2 \  3 \  4 |  5 \  5 |  6 \  6 | 
    \end{minted}
    
    
\item \textbf{\textcolor{blue}{Write all the parenthesizations of $ABCDE$.
Associate them in a natural way with (setting $n=5$) the
terms $P(i)P(n-i)$, $i=1,2,3,4$ given in the recursion for $P(n)$.}}
    \\ Splitting 1 - 4 gives $P(1)P(4) = 5$ parenthesizations:
    \\ $A(B(C(DE)))$
    \\ $A(B((CD)E))$
    \\ $A((BC)(DE))$
    \\ $A((B(CD))E)$
    \\ $A(((BC)D)E)$
    \\ Splitting 4 - 1 gives $P(4)P(1) = 5$ parenthesizations:
    \\ $(A(B(CD)))E$
    \\ $(A((BC)D))E$
    \\ $((AB)(CD))E$
    \\ $(((AB)C)D)E$
    \\ $((A(BC))D)E$
    \\ Splitting 2 - 3 gives $P(2)P(3) = 2$ parenthesizations:
    \\ $(AB((CD)E)$
    \\ $(AB)(C(DE))$
    \\ Splitting 3 - 2 gives $P(3)P(2) = 2$ parenthesizations:
    \\ $((AB)C)(DE)$
    \\ $(A(BC))(DE)$

    
\item \textbf{\textcolor{blue}{Let $x_1,\ldots,x_m$ be a sequence of distinct real numbers. For $1\leq i\leq m$ let $INC[i]$ denote the length of the longest
increasing subsequence ending with $x_i$.  Let $DEC[i]$ denote the 
length of the longest decreasing subsequence ending with $x_i$.  
{\tt Caution:} The subsequence must {\em use} $x_i$.  For example,
$20,30,4,50, 10$.  Now $INC[5]=2$ because of $4,10$ -- we do
{\em not} count $20,30,50$.}}
    \\ \textbf{DYNAMIC PROGRAMMING = Optimal substructures + overlapping subproblems}
    \begin{enumerate}
    \item \textbf{\textcolor{blue}{Find an efficient method for finding the values $INC[i]$, $1\leq i\leq n$. (You should find $INC[i]$ based on the previously found $INC[j]$, $1\leq j< i$.  Your algorithm should take time $O(i)$ for each
    particular $i$ and thus $O(n^2)$ overall.)}}
        \\ We are searching for INC[i] which is the \textbf{length} of the longest increasing subsequence ending with $x_i$.
        \begin{itemize}
            \item If there is no $x_j$ such that $0 < j < i$ and $x_j < x_i$, then INC[i] = 1
            \item Else, INC[i] = 1 + max(INC[j])
        \end{itemize}
        \\ Each INC[i] takes $O(n)$ so the total time is $O(n^2)$. The algorithm has been implemented in Python:
        \begin{minted}{python}
def INC(A, i):
    """ Length of increasing subsequence in A ending at A[i] """
    for j in range(i - 1, -1, -1):
        if A[j] < A[i]:
            return 1 + INC(A, j)
    return 1

def DEC(A, i):
    """ Length of decreasing subsequence in A ending at A[i] """
    for j in range(i - 1, -1, -1):
        if A[j] > A[i]:
            return 1 + DEC(A, j)
    return 1

if __name__ == '__main__':
    print INC([7,5,3,9,6,7], 3) # 2
    print DEC([7,5,3,9,6,7], 3) # 1
        \end{minted}

    \item \textbf{\textcolor{blue}{Let $LIS$ denote the length of the longest increasing subsequence of $x_1,\ldots,x_m$.  Show how to find $LIS$ from
    the values $INC[i]$. Your algorithm, {\em starting with} the $INC[i]$,
    should take time $O(n)$. Similarly, let $DIS$ denote the length of the longest
    decreasing subsequence of $x_1,\ldots,x_m$.  Show how to find $DIS$ from
    the values $DEC[i]$. }}
        \\ $LIS =$ maximum of all $INC[i]$ and $DIS =$ maximum of all $DEC[i]$.
        \\ Using the Python code, above, these two were implemented in Python:
        \begin{minted}{python}
def LIS(A):
    return max([INC(A, i) for i in range(len(A))])

def DIS(A):
    return max([DEC(A, i) for i in range(len(A))])

if __name__ == '__main__':
    print LIS([7,5,3,9,6,2]) # 2
    print DIS([7,5,3,9,6,2]) # 3
        \end{minted}
        
    \item \textbf{\textcolor{blue}{Suppose $i<j$.  {\em Prove} that it is impossible to have $INC[i]=INC[j]$ {\em and} $DEC[i]=DEC[j]$.  ({\tt Hint:} Show that {\em if} $x_i < x_j$ then $INC[j] \geq INC[i] + 1$.)}}
        \\ $x_i < x_j \Rightarrow INC[j] \geq INC[i] + 1$ since $x_j$ can be appended to the maximal increasing sequence ending with $x_i$. 
        \\ $x_i > x_j \Rightarrow DEC[j] \geq DEC[i] + 1$ since $x_j$ can be appended to the maximal decreasing sequence ending with $x_i$. 
    \item \textbf{\textcolor{blue}{Deduce the following celebrated result (called the Monotone Subsequence Theorem) of Paul Erd\H{o}s and George Szekeres:  Let $m=ab+1$.  Then any sequence $x_1,\ldots,x_m$ of distinct real numbers either $LIS>a$ or $DIS>b$.  (Idea: Assume not and look at the pairs $(INC[i],DEC[i])$.)  Paul Erd\H{o}s was a great twentieth century mathematician,  whose work remains highly influential in many areas.}}
        \\ $\left\{
                \begin{array}{ll}
                    LIS \leq a \\
                    DIS \leq b
                \end{array}
            \right. \Rightarrow \exist \mbox{ only } ab \mbox{ possibilities for the pair }(INC[i], DEC[i])$
        \\ However, for the previous part, we obtained $ab + 1$ distinct pairs.
        
        
        
    \end{enumerate}
    
    
\item \textbf{\textcolor{blue}{Find an optimal parenthesization of a matrix-chain product whose sequence of dimensions is $5,10,3,12,5,50,6$.}}
    \\ The matrix chain product of $A_1A_2A_3...A_n$ can be broken to $(A_1...A_k)(A_{k+1}...A_n)$.
    \\ There are two matrices:
    \begin{itemize}
        \item s to record the minimum number of operations
        \item m to record the parenthesization
    \end{itemize}
    The algorithm is as follows, where $p$ is an array containing the matrix dimensions such that $A_i$ has the dimensions $p_{i-1}, p_i$.
    \begin{algorithm}
        \SetKwFunction{matrixchainorder}{MATRIX-CHAIN-ORDER}
        \Indm\matrixchainorder{p}\\
        \Indp
            n = p.length - 1 \\
            Let m[1..n, 1..n] and s[1..n-1, 2..n] be matrices \\
            \For{i = 1 to n}{
                m[i, i] = 0 \\
            }
            \For{l = 2 to n}{
                \For{i = 1 to n - l + 1}{
                    j = i + l - 1 \\
                    m[i, j] = $\infty$ \\
                    \For{k = i to j - 1}{
                        q = m[i, k] + m[k+1, j] + $p_{i-1}p_kp_j$ \\
                        \If{q $<$ m[i, j]}{
                            m[i, j] = q \\
                            s[i, j] = k \\
                        }
                        
                    }
                }
            }
            \Return{m and s}
    \end{algorithm}
    \\ This algorithm was then implemented in Python (headaches for the 0-based index): \\
    \begin{minted}{python}
# Matrix Ai has dimension p[i-1] x p[i] for i = 1..n
def matrixchainorder(p):
    n = len(p) - 1
    m = [[0 for _ in range(n)] for _ in range(n)]
    s = [[0 for _ in range(n)] for _ in range(n)]
    for i in range(n):
        m[i][i] = 0
    for l in range(1, n): # l is chain length
        for i in range(0, n-l):
            j = i + l # max is n-(n-1)-1 + n - 1 = n-1 
            m[i][j] = float("inf")
            for k in range(i, j):
                #q = cost/scalar mutliplications
                q = m[i][k] + m[k+1][j] + p[i]*p[k+1]*p[j+1]
                if q < m[i][j]: #always the case here
                    m[i][j] = q
                    s[i][j] = k + 1
    return m, s

    


if __name__ == '__main__':
    m, s = matrixchainorder([5, 10, 3, 12, 5, 50, 6])
    print_matrix(m)
    print_matrix(s)
    \end{minted}
    \\\\ This program was hence printing m and s as follows:
    \begin{itemize}
        \item The matrix m is printed as:
        \begin{tabular}{ c c c c c c }
        0 & 150 & 330 & 405 & 1655 & 2010 \\
        0 & 0 & 360 & 330 & 2430 & 1950 \\
        0 & 0 & 0 & 180 & 930 & 1770 \\
        0 & 0 & 0 & 0 & 3000 & 1860 \\
        0 & 0 & 0 & 0 & 0 & 1500 \\
        0 & 0 & 0 & 0 & 0 & 0 \\
        \end{tabular}
        \item The matrix s is printed as:
        \begin{tabular}{ c c c c c c }
        0 & 1 & 2 & 2 & 4 & 2 \\
        0 & 0 & 2 & 2 & 2 & 2 \\
        0 & 0 & 0 & 3 & 4 & 4 \\
        0 & 0 & 0 & 0 & 4 & 4 \\
        0 & 0 & 0 & 0 & 0 & 5 \\
        0 & 0 & 0 & 0 & 0 & 0 \\
        \end{tabular}
    \end{itemize}
    \\ Now to do this manually:
    \begin{itemize}
        \item Start with the diagonal set to zeroes (m[1,1]=m[2,2]=...=0)
        \item Start at m[1,2] = $p_0 \times p_1 \times p_2 = 5 \times 10 \times 3 = 150$ then go down diagonally
        \item m[2,3] = $p_1 \times p_2 \times p_3 = 10 \times 3 \times 12 = 360$
        \item m[3,4] = $p_2 \times p_3 \times p_4 = 3 \times 12 \times 5 = 180$
        \item m[4,5] = $p_3 \times p_4 \times p_5 = 12 \times 5 \times 50 = 3000$
        \item m[5,6] = $p_4 \times p_5 \times p_6 = 5 \times 50 \times 6 = 1500$
    \end{itemize}
    We now have 
    \begin{tabular}{ c c c c c c }
    0 & 150 & x & x & x & x \\
    0 & 0 & 360 & x & x & x \\
    0 & 0 & 0 & 180 & x & x \\
    0 & 0 & 0 & 0 & 3000 & x \\
    0 & 0 & 0 & 0 & 0 & 1500 \\
    0 & 0 & 0 & 0 & 0 & 0 \\
    \end{tabular}
    We then calculate using the recurrence relation $m[i,j] = min(m[i,k] + m[k+1, j] + p_{i-1}p_kp_j \mbox{ for } k \mbox{ from } i \mbox{ to } j)$ if $i < j$.
    \begin{itemize}
        \item m[1,3]
        \begin{itemize}
            \item At $k = 1$, $m[1,3] = min(m[1,1] + m[2, 3] + p_{0}p_1p_3) = 0+360+600 = 960$
            \item At $k = 2$, $m[1,3] = min(m[1,2] + m[3, 3] + p_{0}p_2p_3) = 150+0+180 = 330$
        \end{itemize}
        \\ Therefore $m[1,3] = 330$.
        \\ Etc.
    \end{itemize}
    \textcolor{red}{Conclusion missing !!!}
    
    
    
\item \textbf{\textcolor{blue}{Some exercises in logarithms:}}
    \\ \textbf{NOTE: } $\lg(x) = \frac{\log(x)}{\log(2)}$ for BST and in binary context!
    \begin{enumerate}
    \item \textbf{\textcolor{blue}{Write $\lg(4^n/\sqrt{n})$ in simplest form.  What is its asymptotic value.}}
        \\ $\lg(4^n/\sqrt{n}) = \lg(4^n) - \lg(n^{1/2}) = n\lg(4) - \frac{1}{2}\lg(n) = 2n - \frac{\lg(n)}{2}$ so its asymptotic value is infinity as $n$ dominates $\lg(n)$.
    \item \textbf{\textcolor{blue}{Which is bigger, $5^{313340}$ or $7^{271251}$? Give reason. (You can use a calculator but you can't use any numbers bigger than $10^9$.)}}
        \\ Suppose $5^{313340}$ is lower than $7^{271251}$:
        \\ $5^{313340} < 7^{271251} \Rightarrow 313340\lg(5) < 271251\lg(7) \Rightarrow 727552.9492 < 761497.8299$
        \\ Our assumption was correct, so $7^{271251}$ is bigger.
    \item \textbf{\textcolor{blue}{Simplify $n^2\lg(n^2)$ and $\lg^2(n^3)$.}}
        \\ $n^2\lg(n^2) = 2n^2\lg(n)$
        \\ $\lg^2(n^3) = (3\lg(n))(3\lg(n)) = 9\lg^2(n)$
    \item \textbf{\textcolor{blue}{Solve (for $x$) the equation $e^{-x^2/2}=\frac{1}{n}$.}}
        \\ $e^{-x^2/2}=\frac{1}{n} \Rightarrow 1=\frac{e^{x^2/2}}{n} \Rightarrow \ln(1)=\ln(\frac{e^{x^2/2}}{n}) \Rightarrow 0 = \ln(e^{x^2/2}) - \ln(n) \Rightarrow 0 = \frac{x^2}{2} - \ln(n) \Rightarrow 2\ln(n) = x^2 \Rightarrow x = \mp \sqrt{2\ln(n)}$
    \item \textbf{\textcolor{blue}{Write $\log_n(2^n)$ and $\log_nn^2$ in simple form.}}
        \\ $\log_n(2^n) = \frac{\log(2^n)}{\log(n)} = \frac{n\log(2)}{\log(n)} = \frac{n}{\lg(n)}$
        \\ $\log_nn^2 = \frac{\log(n^2)}{\log(n)} = \frac{2\log(n)}{\log(n)} = 2$
    \item \textbf{\textcolor{blue}{What is the relationship between $\lg n$ and $\log_3n$?}}
        \\ $\lg(n) = \frac{\log(n)}{\log(2)}$ and $\log_3(n) = \frac{\log(n)}{\log(3)}$.
        \\ Now $\frac{\log(3)}{\log(2)} \approx 1.585$ so $\lg(n)$ is $\approx$ 1.58 times the value of $\log_3(n)$.
    \item \textbf{\textcolor{blue}{Assume $i< n$.  How many times need $i$ be doubled before it reaches (or exceeds) $n$?}}
        \\ Let $x$ be the number of times we need to double i.
        \\ We need $i \times 2^x \geq n \Rightarrow x \geq \lg(\frac{n}{i})$ so we need x to be the \textbf{ceiling} of $\lg(\frac{n}{i})$.
    \item \textbf{\textcolor{blue}{Write $\lg[n^ne^{-n}\sqrt{2\pi n}]$ precisely as a sum in simplest form.  What is it asymptotic to as $n\ra \infty$?  What is interesting about the bracketed expression?}}
        \\ $\lg[n^ne^{-n}\sqrt{2\pi n}] = \lg(n^n) + \lg(e^{-n}) + \lg(\sqrt{2\pi n}) = n\lg(n) - n\lg(e) + 0.5\lg(2\pi n) = n\lg(n) - n\lg(e) + 0.5\lg(2\pi) + 0.5\lg(n)$. This is a \textbf{Stirling's formula} and is asymptotic to $n!$
    \end{enumerate}
\end{enumerate}

\begin{quote}
There is a theory which states that if ever anybody discovers exactly what
the Universe is for and why it is here, it will instantly disappear and be
replaced by something even more bizarre and inexplicable. There is another
theory which states that this has already happened.
\\ Douglas Adams
\end{quote}

\end{document}