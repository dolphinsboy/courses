\documentclass[11pt]{article}
\pagestyle{empty}
\usepackage{color}
\usepackage{forest}
\usepackage[options]{algorithm2e}
\usepackage{minted}
\usepackage{fancyhdr}
\usepackage{lastpage}
\pagestyle{fancy}
\renewcommand{\headrulewidth}{0pt}
\cfoot[R]{\thepage~of~\pageref{LastPage}}
\addtolength{\oddsidemargin}{-.875in}
\addtolength{\evensidemargin}{-.875in}
\addtolength{\textwidth}{1.75in}
\addtolength{\topmargin}{-.875in}
\addtolength{\textheight}{1.75in}

\begin{document}
\begin{center} {\Large\bf FA, Homework 1}  \\ Quentin McGaw (qm301) \\ 02/02/17
\end{center}

\begin{enumerate}
\item \textbf{\textcolor{blue}{Let $A$ be a max-heap with heapsize fifty million, being used as a priority queue.  Suppose {\tt HEAP-INCREASE-KEY(A,300,key)} is called. What is the minimum and maximum number of exchanges that can take place?}}
    \begin{itemize}
        \item The minimum number of exchanges is $0$ if $key < A[300/2]$ (parent's key).
        \item The maximum number of exchanges occurs when $key$ is the largest key in the heap. There would then be exchanges with $A[150]$, $A[75]$, $A[37]$, $A[18]$, $A[9]$, $A[4]$, $A[2]$ and $A[1]$. This is a total of $8$ exchanges. \textbf{Note} that this is $\left \lfloor{\log_2(300)}\right \rfloor = \left \lfloor{\lg(300)}\right \rfloor = 8$.
    \end{itemize}
    
\item \textbf{\textcolor{blue}{When $A$ is an array with length fifty million and {\tt MAX-HEAPIFY(A,300)} is called. What is the minimum and maximum number of exchanges that can take place?}}
    \begin{itemize}
        \item The minimum number of exchanges is $0$ if the key at $300$ is bigger than the keys of both its children.
        \item The maximum number of exchanges occurs when we have to work all the way down to the leaves. We would have to exchange (Worst case) with $A[600]$, $A[1200]$, $A[2400]$, $A[4800]$, $A[9600]$, $A[19200]$, $A[38400]$, $A[76800]$, $A[153600]$, $A[307200]$, $A[614400]$, $A[1228800]$, $A[2457600]$, $A[4915200]$, $A[9830400]$, $A[19660800]$ and $A[39321600]$, so a total of $17$ exchanges (We chose the left side as indexes are smaller than on the right side). \textbf{Note} that this is $\left \lfloor{\lg(\frac{n}{i})}\right \rfloor = \left \lfloor{\lg(\frac{50000000}{300})}\right \rfloor = 17$.
    \end{itemize}

\item \textbf{\textcolor{blue}{Consider a min-heap $H$ with length $1023$ \footnote{Did you recognize $1023$ as a special number? It's one less than $1024=2^{10}$. The binary tree with that many nodes just fills out a row!}. Assume the elements of the array are distinct. Let $x$ be the third smallest element in the array.}}
\\ \textbf{\textcolor{blue}{What are the possible positions for $x$.}}
    \\ $x$ can be at depth 1 or 2, so from position 2 to position 7 in the array.
\\ \textbf{\textcolor{blue}{Let $y=H[700]$. Can $y$ be the largest element in the array?}}
    \\ As $y$ is at a position more than $\frac{1023}{2}$, it is therefore a leaf. In a min-heap, any leaf can be the largest element.
\\ \textbf{\textcolor{blue}{Can $y$ be the smallest element in the array?}}
    \\ No because the smallest in a min-heap is the root.
\\ \textbf{\textcolor{blue}{Give all $i$ for which it is possible that $y$ is the $i$-th smallest element of the array.}}
    \\ Its ancestors are ($350,175,87,43,21,10,5,2,1$) and are smaller than $y$ so $y$ can be at best $10^{th}$ smallest. At worst, it will be the largest element and thus the $1023^{th}$ smallest element. Therefore $10 \leq i \leq 1023$.

\item \textbf{\textcolor{blue}{Using the figures in the text as a model, illustrate the operation of
{\tt BUILD-MAX-HEAP} on the array $A=(5,3,17,10,84,19,6,22,9)$}}
    \\
    \begin{forest}
    for tree={circle,draw, l sep=20pt}
    [5
        [3
            [10
                [22]
                [9]
            ]
            [84]
        ]
        [17
            [19]
            [6]
        ]
    ]
    \end{forest}
    \begin{forest}
    for tree={circle,draw, l sep=20pt}
    [5
        [3
            [22
                [10]
                [9]
            ]
            [84]
        ]
        [17
            [19]
            [6]
        ]
    ]
    \end{forest}
    \begin{forest}
    for tree={circle,draw, l sep=20pt}
    [5
        [3
            [22
                [10]
                [9]
            ]
            [84]
        ]
        [19
            [17]
            [6]
        ]
    ]
    \end{forest}
    \begin{forest}
    for tree={circle,draw, l sep=20pt}
    [5
        [84
            [22
                [10]
                [9]
            ]
            [3]
        ]
        [19
            [17]
            [6]
        ]
    ]
    \end{forest}
    \begin{forest}
    for tree={circle,draw, l sep=20pt}
    [84
        [5
            [22
                [10]
                [9]
            ]
            [3]
        ]
        [19
            [17]
            [6]
        ]
    ]
    \end{forest}
    \begin{forest}
    for tree={circle,draw, l sep=20pt}
    [84
        [22
            [5
                [10]
                [9]
            ]
            [3]
        ]
        [19
            [17]
            [6]
        ]
    ]
    \end{forest}
    \begin{forest}
    for tree={circle,draw, l sep=20pt}
    [84
        [22
            [10
                [5]
                [9]
            ]
            [3]
        ]
        [19
            [17]
            [6]
        ]
    ]
    \end{forest}

\item \textbf{\textcolor{blue}{The operation {\tt HEAP-DELETE(A,t)} deletes the item in node $t$
from heap $A$.  Give an implementation of {\tt HEAP-DELETE} that runs in
$O(\lg n)$ time for an $n$-element max-heap.}}
    \\
    \begin{algorithm}[H]
        \SetKwFunction{heapdel}{HEAP-DELETE}
        \Indm\heapdel{A, t}\\
        \Indp
            \eIf{t = A.heapsize}{
                A.heapsize$--$ \\            
            }{
                A[t] = A[A.heapsize] \\
                A.heapsize$--$ \\
            }
            \eIf{A[t] $>$ A[$\pi[t]$]}{
                \While{$\exist \pi[t]$}{ \tcp{While t is not the root}
                    A[t], A[$\pi[t]$] = A[$\pi[t]$], A[t] \\
                    t = $\pi[t]$
                }
            }{
                MAX-HEAPIFY(A, t) \tcp{A[t] is too small !}
            }
        \caption{HEAP-DELETE, where A is the heap and t is the node to be deleted}
    \end{algorithm}
    First a simple case: If $t$ is $A.heapsize$ then simply decrement
    $A.heapsize$.  ELSE first reset $A[t]\leftarrow A[A.heapsize]$ and
    decrement $A.heapsize$.  Now you have the right elements but $A[t]$
    may be in the wrong places.  First check if $A[t]> A[parent[t]]$
    (Ignore this if $t$ is the root.)  If so, we have a WHILE loop,
    exchanging $A[t]$ and $A[parent[t]]$ and then resetting $t\leftarrow parent[t]$
    while $t$ is not the root and $A[t]> A[parent[t]]$. ELSE (that is, if
    we did not have $A[t]> A[parent[t]]$, then the only (possible) problem
    is that $A[t]$ is too small.  So we apply $MAX-HEAPIFY[A,t]$.

\item \textbf{\textcolor{blue}{Let $A$ be an array of length $127$ in which the values are distinct and in increasing order. In the procedure {\tt BUILD-MAX-HEAP(A)} precisely how many times will two elements of the array be exchanged?}}
    \\ BUILD-MAX-HEAP(A) essentially works as in the following Python code:
    \begin{minted}{python}
    def max_heapify(self, A, i):
        """ Outputs A modified so that i roots as heap
            Running time O(log n) where n = len(A) - i
            
            Args:
                A (list): Left and right children of i root heaps (but i may not)
                i (int): Array index
        """
        left = 2*i + 1
        right = 2*i + 2
        largest = i
        if left < len(A) and A[left] > A[largest]:
            largest = left
        if right < len(A) and A[right] > A[largest]:
            largest = right
        if largest != i:
            A[i], A[largest] = A[largest], A[i]
            self.max_heapify(A, largest)

    def build_max_heap(self, A):
        """ Outputs A modified to represent a heap
            Running time O(n) where n = len(A)
            
            Args:
                A (list): Unsorted array
                i (int): Array index
        """
        for i in range(len(A)/2, -1, -1):
            self.max_heapify(A, i)
    
    \end{minted}
    \begin{itemize}
        \item BUILD-MAX-HEAP(A) starts from i = LENGTH(A)/2 = 63 down to i = 1
        \item At every step, MAX-HEAPIFY(A, i) is executed.
        \item For $32 \leq i \leq 63$, there is 1 exchange
        \item For $16 \leq i \leq 31$, there are 2 exchanges
        \item For $8 \leq i \leq 15$, there are 3 exchanges
        \item For $4 \leq i \leq 7$, there are 4 exchanges
        \item For $2 \leq i \leq 3$, there are 5 exchanges
        \item For $i = 1$ (the root), it goes to the bottom so there are 6 exchanges
    \end{itemize}
    Total: $32 \times 1 + 16 \times 2 + 8 \times 3 + 4 \times 4 + 2 \times 5 + 1 \times 6 = 120$ exchanges

\item \textbf{\textcolor{blue}{Now suppose the values are distinct and in decreasing order. Again, in the procedure {\tt BUILD-MAX-HEAP(A)} precisely how many times will two elements of the array be exchanged?}}
    \\ Never!  Each element will be placed originally in precisely its correct final spot.
\end{enumerate}
    
\end{document}