\documentclass[11pt]{article}
\pagestyle{empty}
\usepackage{color}
\usepackage{forest}
\usepackage{listings}
\addtolength{\oddsidemargin}{-.875in}
\addtolength{\evensidemargin}{-.875in}
\addtolength{\textwidth}{1.75in}
\addtolength{\topmargin}{-.875in}
\addtolength{\textheight}{1.75in}

\begin{document}
\begin{center} {\Large\bf FA, Homework 1}  \\ Quentin McGaw (qm301) \\ 01/23/17
\end{center}

\begin{enumerate}
\item \textbf{\textcolor{blue}{Let $A$ be a max-heap with heapsize fifty 
million, being used as a priority queue.  Suppose 
{\tt HEAP-INCREASE-KEY(A,300,key)} is called. What is the minimum & maximum number 
of exchanges that can take place?}}
    \\ The minimal number is zero, if the new key has smaller value than
    the key of the parent of $300$.  The maximal number is achieved when
    the new key is the largest key in the heap, and so the exchanges go
    up to the root.  So there are exchanges with $150$, then $75$, then
    $37$, then $18$, then $9$, then $4$, then $2$, then $1$ for a total
    of eight exchanges.  (In general the position $A[i]$ is on row
    $j$ (counting the root as row zero)
    where $2^j\leq i < 2^{j+1}$.  Reversing, $j=\lfloor \lg i\rfloor$.
    The maximum number of exchanges would then be $j$.

\item \textbf{\textcolor{blue}{When $A$ is an array with length fifty 
million and {\tt MAX-HEAPIFY(A,300)} is called. What is the minimum & maximum 
number of exchanges that can take place?}}
    \\ With {\tt MAX-HEAPIFY} we work {\em down} to the leaves.  The minimum
    is zero, if the key at $300$ is bigger than the keys of both its children.
    The maximum would be if we have to work all the way down the the leaves.
    We could exchange with $600,1200,2400,4800,9600,19200,38400$ 
    then $76800,153600,307200,614400$ then (in rough millions) $1.3,2.6$,$5.2$,
    $10.4, 20.8,41.6$.  So the total would be seventeen exchanges.  (One might also 
    exchange with the right children, but as we are examining worst case we
    consider the left children as they have smaller indexes.)  In general, if
    the array has length $n$ then starting at $i$ there is a descendent $j$
    generations down if and only if $i2^j\leq n$, or, equivalently, $j \leq \lg(n/i)$.
    So the maximal number would be $\lfloor \lg(n/i)\rfloor$.

\item \textbf{\textcolor{blue}{Consider a min-heap $H$ with length $1023$
\footnote{Did you recognize $1023$ as a special number? It's one less 
than $1024=2^{10}$. The binary tree with that many nodes just fills 
out a row!}.
Assume the elements of the array are distinct. Let $x$ be the third 
smallest element in the array.}}
\\ \textbf{\textcolor{blue}{What are the possible positions for $x$.}}
    \\ It could be in any position in the first two rows, that is, from
    two to seven.
\\ \textbf{\textcolor{blue}{Let $y=H[700]$. Can $y$ be the largest element in the array?}}
    \\ Sure, any leaf could be the largest in a min-heap.
\\ \textbf{\textcolor{blue}{Can $y$ be the smallest element in the array?}}
    \\ No way, the smallest in a min-heap must be at the root.
\\ \textbf{\textcolor{blue}{Give all $i$ for which it is possible that $y$ 
                            is the $i$-th smallest element of the array.}}
    \\ Its ancestors ($350,175,87,43,21,10,5,2,1$) must be smaller
       so it can be at best tenth smallest.  Its descendents must be
       bigger but it doesn't have any descendents.  A full proof is
       a bit complicated (did you find it?!) but these are the only
       conditions and it could be the $i$-th smallest for any 
       $10\leq i\leq 1023$.

\item \textbf{\textcolor{blue}{Using the figures in the text as a model, illustrate the operation of
{\tt BUILD-MAX-HEAP} on the array $A=(5,3,17,10,84,19,6,22,9)$}}

\item \textbf{\textcolor{blue}{The operation {\tt HEAP-DELETE(A,t)} deletes the item in node $t$
from heap $A$.  Give an implementation of {\tt HEAP-DELETE} that runs in
$O(\lg n)$ time for an $n$-element max-heap.}}
    \\ First a simple case: If $t$ is $A.heapsize$ then simply decrement
    $A.heapsize$.  ELSE first reset $A[t]\leftarrow A[A.heapsize]$ and
    decrement $A.heapsize$.  Now you have the right elements but $A[t]$
    may be in the wrong places.  First check if $A[t]> A[parent[t]]$
    (Ignore this if $t$ is the root.)  If so, we have a WHILE loop,
    exchanging $A[t]$ and $A[parent[t]]$ and then resetting $t\leftarrow parent[t]$
    while $t$ is not the root and $A[t]> A[parent[t]]$. ELSE (that is, if
    we did not have $A[t]> A[parent[t]]$, then the only (possible) problem
    is that $A[t]$ is too small.  So we apply $MAX-HEAPIFY[A,t]$.

\item \textbf{\textcolor{blue}{Let $A$ be an array of length $127$ in which the values are
distinct and in increasing order.  In the procedure 
{\tt BUILD-MAX-HEAP(A)} precisely how many times will two elements
of the array be exchanged?}}
    \\ BUILD-MAX-HEAP(A) starts from I = LENGTH(A)/2 DOWN to 1, every I will 
    do Max-Heapify. \\ For $32\leq I\leq 63$ ,there should be one exchange.
    \\ For $16\leq I \leq 31$, there should be 2 exchanges. 
    \\ For $8\leq  I \leq 17$, there should be 3 exchanges. 
    \\ For $I=4,5,6,7$ , there should be 4 exchanges. 
    \\ For $I = 2,3$ there should be 5 exchanges. 
    \\ The root goes down to the bottom, 6 exchanges. 
    \\ Total: $32\cdot 1+16\cdot 2+8\cdot 3+4\cdot 4+2\cdot 5+1\cdot 6 = 120$
\item \textbf{\textcolor{blue}{Now suppose the values are distinct and in decreasing order. Again, in the procedure 
{\tt BUILD-MAX-HEAP(A)} precisely how many times will two elements
of the array be exchanged?}}
    \\ Never!  Each element will be placed originally in precisely its
    correct final spot.
\end{enumerate}
    
\end{document}

