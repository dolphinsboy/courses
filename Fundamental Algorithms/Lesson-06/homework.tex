\documentclass[11pt]{article}
\pagestyle{empty}
\usepackage{color}
\usepackage{forest}
\usepackage{listings}
\usepackage{minted}
\usepackage{fancyhdr}
\usepackage{lastpage}
\usepackage[options]{algorithm2e}
\pagestyle{fancy}
\renewcommand{\headrulewidth}{0pt}
\cfoot[R]{\thepage~of~\pageref{LastPage}}
\addtolength{\oddsidemargin}{-.875in}
\addtolength{\evensidemargin}{-.875in}
\addtolength{\textwidth}{1.75in}
\addtolength{\topmargin}{-.875in}
\addtolength{\textheight}{1.75in}

\begin{document}
\begin{center} {\Large\bf FA, Homework 6}  \\ Quentin McGaw (qm301) \\ 03/09/17
\end{center}

\begin{quote}
I am slow to learn and slow to forget that which I have learned.  My
mind is like a piece of steel; very hard to scratch anything on it
and almost impossible after you get it there to rub it out.
\\ -- Abraham Lincoln
\end{quote}

\begin{enumerate}
\item \textbf{\textcolor{blue}{(Continuation of Problem from Last Assignment)
Consider a Binary Search Tree $T$ with vertices $a,b,c,d,e,f,g,h$
and $ROOT[T]=a$ and with the following values ($N$ means NIL)
\begin{center}
\begin{tabular}{r|rrrrrrrr} 
vertex & a&b&c&d&e&f&g&h \\
parent & N&e&e&a&d&g&c&a  \\
left & h & N & N & e & c & N & f & N \\
right & d & N & g & N & b & N & N & N
\end{tabular}
\end{center}}}
    \\ We have the following tree:
    \\\\
    \begin{center}
    \begin{forest}
    for tree={circle,draw, l sep=20pt}
    [
    [80,edge label={node[midway,left] {a}}
        [70,edge label={node[midway,left] {h}}]
        [200,edge label={node[midway,right] {d}}
          [150,edge label={node[midway,left] {e}}
            [140,edge label={node[midway,left] {c}}
              [,phantom]
              [148,edge label={node[midway,right] {g}}
                [143,edge label={node[midway,left] {f}}]
                [,phantom]
              ]
            ]
            [170,edge label={node[midway,right] {b}}]
          ]
          [,phantom]
        ]
      ] 
    ]
    ]
    \end{forest}
    \end{center}

    \begin{enumerate}
    \item \textbf{\textcolor{blue}{Which is the successor of $c$.  
    Illustrate how the program {\tt SUCCESSOR} will find it.}}
        \\ We want to find the successor of node $150$ on its left side (vertex $x$). We therefore search 
        for the node with value $150$ and its left child is the successor, otherwise there is no successor.
        \\ To do this, the following Python code was implemented with the recursive function SuccessorVerbose.
        \\ Note that you can also implement the SUCCESSOR algorithm in an iterative way or by keeping track of the parent pointer, but this way is neater I think.
        \begin{minted}{python}
class BinarySearchTree:
    class Node(object):
        def __init__(self, key):
            self.val = key
            self.left = None
            self.right = None
            
    def SuccessorVerbose(self, root, parent_key):
        while root.left is not None:
            if root.val == parent_key:
                print "Parent found and has left successor", root.left.val
                return root.left
            elif parent_key < root.val:
                print "Parent's value is on left side of node", root.val
                return self.SuccessorVerbose(root.left, parent_key)
            else:
                print "Parent's value is on right side of node", root.val
                return self.SuccessorVerbose(root.right, parent_key)
        print "No successor found."
        
    def insert(self, root, node):
        if root is None:
            root = node
        else:
            if root.val < node.val:
                if root.right is None:
                    root.right = node
                else:
                    self.insert(root.right, node)
            else:
                if root.left is None:
                    root.left = node
                else:
                    self.insert(root.left, node)
            
bst = BinarySearchTree()
root = bst.Node(80)
bst.insert(root, bst.Node(70))
bst.insert(root, bst.Node(200))
bst.insert(root, bst.Node(150))
bst.insert(root, bst.Node(140))
bst.insert(root, bst.Node(170))
bst.insert(root, bst.Node(148))
bst.insert(root, bst.Node(143))
bst.SuccessorVerbose(root, parent_key=150)
        \end{minted}
        \\\\ The following standard output is thus generated:
        \\ Parent's value is on right side of node 80
        \\ Parent's value is on left side of node 200
        \\ Parent found and has left successor 140
        \\\\ The successor of $c$ is thus the node with value $140$.
        
    \item \textbf{\textcolor{blue}{Which is the minimal element?
    Illustrate how the program {\tt MIN} will find it.}}
        \\ To find the minimum, the MIN program has been implemented in Python 2.7.
        \\ It basically starts at the root and searches for the leftmost node.
        \\ It is implemented in the function minValueVerbose(root).
        \begin{minted}{python}
class BinarySearchTree:
    class Node(object):
        def __init__(self, key):
            self.val = key
            self.left = None
            self.right = None
        
    def minValueVerbose(self, root):
        while root.left is not None:
            print "There is a node on the left of node with value", root.val
            root = root.left    
        print "No more node on the left of node with value", root.val
        print "The minimum is therefore the node with value", root.val
        return root.val
        
    def insert(self, root, node):
        if root is None:
            root = node
        else:
            if root.val < node.val:
                if root.right is None:
                    root.right = node
                else:
                    self.insert(root.right, node)
            else:
                if root.left is None:
                    root.left = node
                else:
                    self.insert(root.left, node)
            
bst = BinarySearchTree()
root = bst.Node(80)
bst.insert(root, bst.Node(70))
bst.insert(root, bst.Node(200))
bst.insert(root, bst.Node(150))
bst.insert(root, bst.Node(140))
bst.insert(root, bst.Node(170))
bst.insert(root, bst.Node(148))
bst.insert(root, bst.Node(143))
bst.minValueVerbose(root)
        \end{minted}
        \\\\\\ The following standard output is hence generated:
        \\ There is a node on the left of node with value 80
        \\ No more node on the left of node with value 70
        \\ The minimum is therefore the node with value 70
        \\
    \item \textbf{\textcolor{blue}{Illustrate the program {\tt DELETE[e]}}}
        \\ The program DELETE[e] searches for the node with the key equal to $e$.
        \\ Once found, there are three cases:
        \\ 1. The node has no childs, in this case it is just removed.
        \\ 2. The node has only one child, which will replace it once it is removed.
        \\ 3. The node has two childs. The minimum child of its right sub-tree will replace it once it is removed.
        \\\\ This program was again implemented in Python 2.7 in the function delete(root, key):
        \begin{minted}{python}
class BinarySearchTree:
    class Node(object):
        def __init__(self, key):
            self.val = key
            self.left = None
            self.right = None
        
    def minValue(self, root):
        while root.left is not None:
            root = root.left
        return root.val
        
    def delete(self, root, key):
        if root is None:
            return None
        if root.val == key:
            if root.left is not None and root.right is not None: # 2 childs
                minValue = self.minValue(root.right)
                root.val = minValue
                root.right = self.delete(root.right, minValue)                    
            elif root.left is not None: # only left child
                return root.left
            elif root.right is not None: # only right child
                return root.right
            else: # no child
                return None
        elif root.val < key:
            root.right = self.delete(root.right, key)
        else:
            root.left = self.delete(root.left, key)
        return root
        
    def insert(self, root, node):
        if root is None:
            root = node
        else:
            if root.val < node.val:
                if root.right is None:
                    root.right = node
                else:
                    self.insert(root.right, node)
            else:
                if root.left is None:
                    root.left = node
                else:
                    self.insert(root.left, node)
                    
    def print_inOrder(self, root):
        if root:
            self.print_inOrder(root.left)
            print root.val,
            self.print_inOrder(root.right)
            
bst = BinarySearchTree()
root = bst.Node(80)
bst.insert(root, bst.Node(70))
bst.insert(root, bst.Node(200))
bst.insert(root, bst.Node(150))
bst.insert(root, bst.Node(140))
bst.insert(root, bst.Node(170))
bst.insert(root, bst.Node(148))
bst.insert(root, bst.Node(143))
bst.print_inOrder(root)
print
bst.delete(root, 140)
bst.print_inOrder(root)
        \end{minted}
        \\\\\\ As you may see, the delete function is recursive and uses the minValue function previously described.
        \\ The standard output generated from running this code is the following:
        \\\\ 70 80 140 143 148 150 170 200
        \\ 70 80 143 148 150 170 200
    \end{enumerate}
    
\item \textbf{\textcolor{blue}{Draw binary search trees of height 
2,3,4,5,6 on the set of keys \{1,4,5,10,16,17,21\}.}}
    \begin{forest}
    for tree={circle,draw, l sep=10pt}
    [10
        [4
            [1]
            [5]
        ]
        [17
            [16]
            [21]
        ]
    ]
    \end{forest}
    \begin{forest}
    for tree={circle,draw, l sep=10pt}
    [10
        [1
            [,phantom]
            [4
                [,phantom]
                [5]
            ]
        ]
        [21
            [16
                [,phantom]
                [17]
            ]
            [,phantom]
        ]
    ]
    \end{forest}
    \begin{forest}
    for tree={circle,draw, l sep=10pt}
    [1
        [,phantom]
        [10
            [5
                [4]
                [,phantom]
            ]
            [21
                [16
                    [,phantom]
                    [17]
                ]
                [,phantom]
            ]
        ]
    ]
    \end{forest}
    \begin{forest}
    for tree={circle,draw, l sep=10pt}
    [1
        [,phantom]
        [5
            [4]
            [10
                [,phantom]
                [16
                    [,phantom]
                    [17
                        [,phantom]
                        [21]
                    ]
                ]
            ]
        ]
    ]
    \end{forest}
    \begin{forest}
    for tree={circle,draw, l sep=10pt}
    [1
        [,phantom]
        [4
            [,phantom]
            [5
                [,phantom]
                [10
                    [,phantom]
                    [16
                        [,phantom]
                        [17
                            [,phantom]
                            [21]
                        ]
                    ]
                ]
            ]
        ]
    ]
    \end{forest}
    \\ \space Height 2 \hfill Height 3 \hfill Height 4 \hfill Height 5 \hfill Height 6

\item \textbf{\textcolor{blue}{What is the difference between 
the binary-search property and the heap property?
(*) Can the heap property be used to print out the keys of 
an $n$-node tree in sorted order in $O(n)$ time? Explain how or why not.}}
    \\ Heap property guarantees elements on higher levels are greater (for max-heap) than elements on lower levels.
    \\ Binary Search property guarantees order, from "left" to "right".
    \\ The heap property can print out the keys of an $n$-node tree in sorted order in $O(n\ log\ n)$.
    \\ This is due to the fact that this property will require us to swap and reheapify each time so 
    it would require $log(n) + log(n-1) + log(n-2) + ... + 1$ which results in $O(n\ log\ n)$.
    
\item \textbf{\textcolor{blue}{You are given an array $A[1\cdots n]$, 
whose values come from a universe $\Omega$. (In application, the values 
would be the keys of records.) You want to test if there are any
duplicates, if there are any $1\leq i < j \leq n$ such that
$A[i]=A[j]$.  You are given a hash function $h:\Omega\ra \{1,\ldots,n\}$
and a table $T[1\cdots n]$ of linked lists, initially all empty.
Using the hash function,  give an algorithm that returns {\tt BAD}
if there is a duplicate and {\tt GOOD} if there is no duplicate.
Discuss the time of the algorithm under the assumption that
calculating the hash function takes unit time.}}
    \\\\ The following algorithm adds elements to the table and check for duplicates.
    \\ If a duplicate is found, the element is not added.
    \\
    \begin{algorithm}[H]
        \For{i from 1 to N}{
            index \leftarrow h(A[i])\\
            head \leftarrow T[index]\\
            \If{head == NULL}{
                head.value \leftarrow A[i]\\
                head.next \leftarrow NULL\\
                \Return "GOOD"\\
            }
            node \leftarrow head\\
            \While{node != NULL}{
                \If{node.value == A[i]}{
                    \Return "BAD"\\
                }
                \If{node.next == NULL}{
                    node.next \leftarrow Node(value=A[i], next=NULL)\\
                    \Return "GOOD"\\
                }
                node \leftarrow node.next\
            }                
        }
        \caption{Duplicate check and hash table algorithm}
    \end{algorithm}
    \\ The outer loop is executed $n$ times, and the inner loop is executed $m$ times where
    $m$ is the number of elements of $A$ hashing to the same index of the table $T$.
    \\ In the worst case, if all the elements of $A$ hash to the same index in $T$, 
    the time required is $(n-1)1 + n = 2n - 1$ which is $O(n)$.
    
    
\item \textbf{\textcolor{blue}{What would the BST tree look like if you 
start with the root $a_1$ with $key[a_1]=1$ (and nothing else)
and then you apply \[ INSERT[a_2],\ldots,INSERT[a_n] \]  in
that order where $key[a_i]=i$ for each $2\leq i\leq n$?  
Suppose the same assumptions of starting with $a_1$ and
the key values but the INSERT commands were done in
{\em reverse} order \[ INSERT[a_n],\ldots,INSERT[a_2] \]}}
    \\ For the first case, the BST tree would then be a straight line from left to right. 
    Its nodes would be $1, 2, 3, ..., n$ from top to bottom.
    It would look like the following tree:
    \\
    \begin{center}
    \begin{forest}
    for tree={circle,draw, l sep=10pt}
    [1
        [,phantom]
        [2
            [,phantom]
            [3
                [,phantom]
                [...
                    [,phantom]
                    [n]
                ]
            ]
        ]
    ]
    \end{forest}
    \end{center}
    \\ In the second case, the root is 1 but we insert elements in the reverse order.
    \\ The BST tree would simply look like the following:
    \\
    \begin{center}
    \begin{forest}
    for tree={circle,draw, l sep=10pt}
    [1
        [,phantom]
        [n
            [n - 1
                [...
                    [3
                        [2]
                        [,phantom]
                    ]
                    [,phantom]
                ]
                [,phantom]
            ]
            [,phantom]
        ]
    ]
    \end{forest}
    \end{center}
    

\end{enumerate}

\begin{quote}
In the novel I never wrote, I wanted the hero to be a computer programmer
because it was the most poetic and romantic occupation I could think
of [...] I conceived of him, whose professional life was spent in the
sanctum of the night (when, I was told, the computers, too valuable to be
unemployed by industry during the day, are free, as it were, to frolic)
devising idioms whereby problems might be fed to the machines and emerge,
under binary percussion, as the music of truth - I conceived of him as
being too fine, transluscent, and scrupulous to live in our coarse age.
He was to be, if the metaphor is biological, an evolutionary abortion,
a mammalian mutation crushed underfoot by dinosours, and, if the metaphor
is mathematical, a hypothetical ultimate, one digit beyond the last real
number.  The title of the book was to be $N+1$.
\\ from {\em The Music School} by John Updike
\end{quote}

\end{document}