\documentclass[11pt]{article}
\pagestyle{empty}
\usepackage{fancyhdr}
\usepackage{lastpage}
\pagestyle{fancy}
\renewcommand{\headrulewidth}{0pt}
\cfoot[R]{\thepage~of~\pageref{LastPage}}
\addtolength{\oddsidemargin}{-.875in}
\addtolength{\evensidemargin}{-.875in}
\addtolength{\textwidth}{1.75in}
\addtolength{\topmargin}{-.875in}
\addtolength{\textheight}{1.75in}

\title{FA Lesson 4}
\begin{document}
\begin{center}
\textbf{FA Lesson 4: Master theorem}
\end{center}
The master theorem \textbf{only} concerns recurrence relations of the form:
\begin{equation}T(n)=aT(\frac{n}{b}) + f(n) \textnormal{ where } a \geq 1 \textnormal{ and } b > 1\end{equation}
\newline $n$ is the size of the problem, $a$ the number of subproblems in recursion, 
$\frac{n}{b}$ the size of each subproblem and $f(n)$ the cost of the work done outside the recursive calls
such as dividing the problem or merging the solutions of the subproblems.

\begin{itemize}
    \item [\textbf{Case 1}] Low overhead
    \begin{equation}
    \left\{
        \begin{array}{ll}
            f(n)=O(n^c) \\
            log_b\ a > c
        \end{array}
    \right. \Rightarrow T(n) = \Theta(n^{log_b\ a})
    \end{equation}
    \item [\textbf{Case 2}] Perfect case
    \begin{equation}
    \left\{
        \begin{array}{ll}
            f(n)=\Theta(n^{c}log^{k}n) \textnormal{ for some constant } k \geq 0 \\
            log_b\ a = c
        \end{array}
    \right. \Rightarrow T(n) = \Theta(n^{c}log^{k+1}\ n)
    \end{equation}
    \item [\textbf{Case 3}] High overhead
    \begin{equation}
    \left\{
        \begin{array}{ll}
            f(n)=\Omega(n^c) \\
            log_b\ a < c \\
            af(\frac{n}{b}) \leq kf(n) \textnormal{ for } k < 1 \textnormal{ and sufficiently large n}
        \end{array}
    \right. \Rightarrow T(n) = \Theta(f(n))
    \end{equation}
\end{itemize}
\newline
\\ Let $S(n) = \frac{T(n)}{n^{log_{b}a}}$
\\ $\Rightarrow S(\frac{n}{b} = \frac{T(\frac{n}{b})}{\frac{n}{b}^{log_{b}a}}$
\\ $\Rightarrow S(\frac{n}{b} = \frac{aT(\frac{n}{b})}{n^{log_{b}a}}$
\\ so $S(n) = S(\frac{n}{b}) + \frac{f(n)}{n^{log_{b}a}}$
\\ $\Rightarrow S(n) = S(\frac{n}{b}) + g(n)$


\end{document}