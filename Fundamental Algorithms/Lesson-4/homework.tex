\documentclass[11pt]{article}
\pagestyle{empty}
\usepackage{color}
\usepackage{minted}
\usepackage[options]{algorithm2e}
\usepackage{fancyhdr}
\usepackage{lastpage}
\pagestyle{fancy}
\renewcommand{\headrulewidth}{0pt}
\cfoot[R]{\thepage~of~\pageref{LastPage}}
\addtolength{\oddsidemargin}{-.875in}
\addtolength{\evensidemargin}{-.875in}
\addtolength{\textwidth}{1.75in}

\addtolength{\topmargin}{-.875in}
\addtolength{\textheight}{1.75in}

\begin{document}
\begin{center} {\Large\bf FA, Homework 4}  \\ Quentin McGaw (qm301) \\ 02/16/17

\end{center}

\begin{quote}
The computer is useless. It can only answer questions.  \\ -- Pablo Picasso
\end{quote}

When asked for the asymptotics answer in a form $\Theta(n^a)$ or
$\Theta(\lg^bn)$ or $\Theta(n^a\lg^bn)$ for some reals $a,b$.

\begin{enumerate}
\item  \textbf{\textcolor{blue}{Consider the recursion $T(n)=9T(n/3) + n^2$ 
with initial value $T(1)=1$.Calculate the {\em precise} values of 
$T(3),T(9),T(27),T(81),T(243)$.}}
        \\\\ $T(3) = 9T(1) + 3^2 = 9(1) + 3^2 = 18 = 2(3^2)$
        \\ $T(9) = 9T(3) + 9^2 = 9(9(1) + 3^2) + 9^2 = 243 = 3(9^2)$
        \\ $T(27) = 9T(9) + 27^2 = 2916 = 4(27^2)$
        \\ $T(81) = 9T(27) + 81^2 = 32805 = 5(81^2)$
        \\ $T(243) = 9T(81) + 243^2 = 354294 = 6(243^2)$
        \\ We used the following Python code to calculate these values:
        \begin{minted}{python}
        def T(n):
            """ Recursive function 
                T(n)=9T(n/3)+n^2 with T(1) = 1
                Only valid for n a power of 3
            """
            if n == 1:
                return 1
            elif n % 3 != 0:
                raise Exception("n should be divisible by 3")
            else:
                return 9*T(n/3) + n*n
        
        if __name__ == "__main__":
            print T(3)
            print T(9)
            print T(27)
            print T(81)
            print T(243)
        \end{minted}
    \textbf{\textcolor{blue}{Make a good (and correct) guess as to the general formula for $T(3^i)$ and
    write this as $T(n)$.  (Don't worry about when $n$ is not a power
    of three.)}}
        \\\\ From our calculations, it seems that $T(3^i) = (i + 1)3^2i$
        \\ We need to express $i$ and $3^2i$ as a function of $n$
        \\ First, $n = 3^i \Rightarrow i = lg_3\ n$
        \\ Also, $n = 3^i \Rightarrow n^2 = 3^{2i}$
        Hence $T(n) = (1 + lg_3\ n)n^2$ \\\\
    \textbf{\textcolor{blue}{Now use the Master Theorem to give, in Thetaland, the asymptotics of $T(n)$.
    Check that the two answers are consistent.}}
        \\\\ In Thetaland, $T(n) = \Theta(n^2 lg\ n)$. With the Master Theorem, we are in the special case because $lg_3\ 9 = 2$ so $T(n) = \Theta(n^2 lg\ n)$ \\

\item \textbf{\textcolor{blue}{Use the Master Theorem to give, in Thetaland, the asymptotics of
these recursions:}}
    \begin{enumerate}
    \item \textbf{\textcolor{blue}{$T(n)= 6T(n/2) + n\sqrt{n}$}}
        \\\\ We can rewrite $T(n)$ as $T(n) = 6T(\frac{n}{2}) + n^{1.5}$, hence we have the following parameters:
        $c = 1.5$, $b = 2$ and $a = 6$.
        \\ Because $log_b\ a = log_2\ 6 = \frac{log\ 6}{log\ 2} = 2.58 > 1.5$, we are in the low overhead case and thus $T(n) = \Theta(n^{log_b\ a}) = \Theta(n^{log_2\ 6})$ \\
    \item \textbf{\textcolor{blue}{$T(n)= 4T(n/2)+n^5$}}
        \\\\ We have the following parameters: $c = 5$, $b = 2$ and $a = 4$.
        \\ Because $log_2\ 4 < 5$, we are in the high overhead case and thus $T(n) = \Theta(f(n)) = \Theta(n^5)$ \\
    \item \textbf{\textcolor{blue}{$T(n)= 4T(n/2)+7n^2+2n+1$}}
        \\\\ We have the following parameters: $c = 2$, $b = 2$ and $a = 4$.
        \\ Because $log_2\ 4 = 2$ and the overhead is $\Theta(n^2)$, we have $T(n) = \Theta(n^2log^{(0+1)}\ n)$ \\
    \end{enumerate}

\item  \textbf{\textcolor{blue}{{\tt Toom-3} is an algorithm similar to the 
Karatsuba algorithm discussed in class.  (Don't worry how
{\tt Toom-3} really works, we just want an analysis given the
information below.)
It multiplies two $n$ digit numbers by making five recursive calls
to multiplication of two $n/3$ digit numbers plus thirty additions
and subtractions.  Each of the additions and subtractions take
time $O(n)$.}}
    \newline
    \textbf{\textcolor{blue}{Give the recursion for the time $T(n)$ for {\tt Toom-3} and
    use the Master Theorem to find the asymptotics of $T(n)$.}}
        \\\\ $T(n) = 5T(\frac{n}{3}) + O(30n) = 5T(\frac{n}{3}) + O(n)$
        \\ From the Master theorem, we have the following parameters: $a = 5$, $b = 3$ and $c = 1$.
        \\ Because $log_b\ a = log_3\ 5 = \frac{log\ 5}{log\ 3} = 1.465 > 1$, we are in the low overhead case and thus $T(n) = \Theta(n^{log_b\ a}) = \Theta(n^{log_3\ 5})$ \\\\
    \textbf{\textcolor{blue}{Compare with the time $\Theta(n^{\log_23})$ of Karatsuba.  Which is faster when $n$
    is large?}}
        \\\\ Because $log_3\ 5 = 1.465 < log_2\ 3 = 1.585$, the $\Theta(n^{log_3\ 5})$ of Toom-3 is faster than the $\Theta(n^{log_2\ 3})$ of Karatsuba, so Toom-3 is faster when n is large. \\

\item \textbf{\textcolor{blue}{Write the following sums in the form $\Theta(g(n))$ with $g(n)$
one of the standard functions.  
In each case give reasonable (they
needn't be optimal) positive $c_1,c_2$ so that the sum is between
$c_1g(n)$ and $c_2g(n)$ for $n$ large.}}
    \begin{enumerate}
    \item \textbf{\textcolor{blue}{$n^2+(n+1)^2+\ldots + (2n)^2$}}
        \\\\ $\Theta(g(n)) = \Theta(n^3)$ and we have $c1 = 0.5$ and $c2 = 1.5$ \\
    \item \textbf{\textcolor{blue}{$\lg^2(1)+\lg^2(2)+\ldots + \lg^2(n)$}}
        \\\\ $\Theta(g(n)) = \Theta(nlg^{2}\ n)$ and we have $c1 = 0.5$ and $c2 = 1.5$ \\
    \item \textbf{\textcolor{blue}{$1^3+\ldots+n^3$.}}
        \\\\ $\Theta(g(n)) = \Theta(n^4)$ and we have $c1 = 0.5$ and $c2 = 1.5$ \\
    \end{enumerate}


\item \textbf{\textcolor{blue}{Give an algorithm for subtracting two $n$-digit decimal numbers.
The numbers will be inputted as $A[0\cdots N]$ and $B[0\cdots N]$
and the output should be $C[0\cdots N]$.  (Assume that the result will
be nonnegative.)}}
    \\ Assuming the digit at position $0$ is the most significant one, the following algorithm works:
    \\\\
    \begin{algorithm}[H]
        \For{$i$ from $0$ to $N$}{
            X = A[i] - B[i]\
            \eIf{X >= 0}{
                C[i] = X\
            }{
                C[i-1]--\
                C[i] = 10 + X\
            }
        }
        \caption{n-digit decimal subtraction algorithm}
    \end{algorithm}
    Note that we used the following Python code to test it out:
    \begin{minted}{python}
    def subtract_n_digit_numbers(A, B):
        N = len(A)
        if N != len(B):
            raise Exception("A and B must have the same length")
        C = [None for _ in range(N)]
        for i in range(0, N):
            x = A[i] - B[i]
            if x >= 0:
                C[i] = x
            else:
                C[i-1] -= 1 #non-negative result so that works
                C[i] = 10 + x
        return C    
    
    if __name__ == "__main__":
        A = [2,9,4,3,2]
        B = [1,5,3,7,1]
        C = subtract_n_digit_numbers(A, B)
        print C
    \end{minted}
    \newline

    \textbf{\textcolor{blue}{How long does your algorithm take, expressing
    your answer in one of the standard $\Theta(g(n))$ forms.}}
        \\\\Assuming the addition and subtraction operations of two digits take $O(1)$, this algorithm takes $\Theta(g(n))=\Theta(n)$ as it is a simple for loop of size $N$.
\end{enumerate}


\begin{quote}
The mind is not a vessel to be filled but a fire to be kindled.  \\ -- Plutarch
\end{quote}

\end{document}

