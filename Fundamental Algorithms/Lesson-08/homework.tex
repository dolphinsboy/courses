\documentclass[11pt]{article}
\pagestyle{empty}
\usepackage{color}
\usepackage{fancyhdr}
\usepackage{lastpage}
\usepackage[options]{algorithm2e}
\pagestyle{fancy}
\renewcommand{\headrulewidth}{0pt}
\cfoot[R]{\thepage~of~\pageref{LastPage}}
\addtolength{\oddsidemargin}{-.875in}
\addtolength{\evensidemargin}{-.875in}
\addtolength{\textwidth}{1.75in}
\addtolength{\topmargin}{-.875in}
\addtolength{\textheight}{1.75in}

\begin{document}
\begin{center} {\Large\bf FA, Homework 8} \\ Quentin McGaw (qm301) \\ 04/06/17
\end{center}

\begin{enumerate}
\item \textbf{\textcolor{blue}{Set $W=\lfloor{\sqrt{N}}\rfloor$. We are {\em given} $PRICE[I]$, $1\leq I\leq W$, the price of a rod of length $I$. Give a program that will output the optimal revenue for a rod of length $N^2$ and give the time, in $\Theta$-land, of the algorithm. Use an auxiliary array $R[J]$, $0\leq J\leq N^2$, where $R[J]$ will give the optimal revenue for a rod of length $J$. You may {\bf not} use the term {\tt MAX} nor {\tt MIN} in your program  Explain, in clear words, how your program is working. (You can and should use {\tt MAX} in your explanations.)}}
    \begin{algorithm}[H]
        Let R = [0..$N^2$] \\
        R[0] = 0 \\
        R[1] = PRICE[1] \\
        \For{J = $2$ to $N^2$}{
            R[j] = 0 \\
            \For{I = 1 to min(J, W)}{
                R[j] = max(R[j], PRICE[I] + R[J-I]) \\
            }
        }
        \Return{R[$N^2$]}
        \caption{Algorithm to find the optimal revenue for a rod of length I}
    \end{algorithm}
    The outer loop executes $1 \leq J \leq N^2$ times.
    \\ The time of the inner loop is min(J,W) and there are thus two ranges:
    \begin{itemize}
        \item $1 \leq J < W$ : J steps and it adds up to $1 + 2 + ... + (W - 1) \approx \frac{W^2}{2} = \Theta(N)$.
        \item $W \leq J \leq N^2$ : W steps and it adds up to W($N^2$ - W) = $\Theta(N^{2.5})$.
    \end{itemize}
    The total time is therefore $\Theta(N + N^{2.5}) = \Theta(N^{2.5})$.
    
\item  \textbf{\textcolor{blue}{Suppose, in the Activity Selector problem, we
instead select the last activity to start that is compatible with all previously
selected activities. Describe how this approach works, write a program for it 
(pseudo-code allowed) and prove that it yields an optimal algorithm.}}
    \\ 
    \begin{algorithm}[H]
        \SetKwFunction{ras}{RECURSIVE-ACTIVITY-SELECTOR}
        \Indm\ras{s, f, k, n}\\
        \Indp
            m = k + 1 \\
            \While{m $\leq$ n and $s_m < f_k$}{
                m$++$ \tcp*{Find the first activity in $S_k$ to finish}
            }
            \eIf{m $\leq$ n}{
                \Return{$\{a_m\}\ \cup $ \ras{s, f, m, n}}
            }{
                \Return{$\emptyset$}
            }
        \caption{Recursive activity selector}
    \end{algorithm}
    In the reverse case, at each point, the last activity to start is selected. There is a recursion down to the single optimal subproblem to find the optimal solution for all remaining activities compatible (proof by induction). \\
    \begin{algorithm}[H]
        \SetKwFunction{rasv}{RECURSIVE-ACTIVITY-SELECTOR-REVERSE}
        \Indm\rasv{s, f, k}\\
        \Indp
            m = k - 1 \\
            \While{m $>$ 0 and $s_k < f_m$}{
                m$--$ \tcp*{Find the last activity in $S_k$ to start}
            }
            \eIf{m $>$ 0}{
                \Return{$\{a_m\}\ \cup $ \rasv{s, f, m}}
            }{
                \Return{$\emptyset$}
            }
        \caption{Recursive activity selector (reverse)}
    \end{algorithm}
    We start with i = n (number of tasks). Consider a nonempty subproblem $S_{ij}$ and let $a_m$ be the activity in $S_{ij}$ with the latest start time $s_m = \max(s_k : a_k \in S_{ij})$.
    \begin{itemize}
        \item $a_m$ is used in a maximum-size subset of mutually compatible activities $S_{ij}$.
        \\ \textbf{Proof: } Suppose $A_{ij}$ is a maximum-size subset of mutually compatible activies of $S_{ij}$. Suppose activities in $A_{ij}$ are ordered in monotonically increasing order of starting time. Let $a_k$ be the last activity of $A_{ij}$. If $a_k = a_m$, we are done since $a_m$ is used in constructing the schedule. Otherwise, we construct $A^{'}_{ij} = A_{ij} - \{a_k\}\ \in \ \{a_m\}$. The activities are disjoint since $a_k$ is the last activity to start and $s_m \geq s_k$. The number of activities in the subset are the same, so it is a maximum-size subset of activities including $a_m$.
        \item The subproblem $S_{mj}$ is empty, so choosing $a_m$ leaves the subproblem $S_{im}$ as the only one that may be nonempty.
        \\ \textbf{Proof: } Suppose $S_{mj}$ is non-empty, so there is an activity $a_k$ such that $f_m \leq s_k < f_k \leq s_j < f_j$. As a consequence, $a_k \in S_{ij}$ which has a later start time than $a_m$, contradicting our initial assumption.
    \end{itemize}
    Let $S$ be the set of activities and $A$ be the maximum-size subset of mutually compatible activities of $S$. We first arrange the activities in order of decreasing start time. Let $a_k = [s_k, f_k)$ be the last activity in $A$. Let $a_m$ be the activity with the earliest finish time. At every step, we discard any overlapping activities with already chosen activities. The latest starting activities remains chosen.
    \\ If $a_k = a_m$ then we found an optimal solution. Otherwise, we build the new set starting with $a_k$ and go up until $a_m \leq a_k$. This set has the same number of activities and is mutually compatible with the original set. The optimal solution for $S$ maps directly to the optimal solution in the new set.    

\item \textbf{\textcolor{blue}{Students (professors too!) often come up with very clever ideas for
optimization programs.  The problem (often!) is that they (sometimes, but that
is enough) give the wrong answer.  Here are three approaches and {\em your} problem,
in each case, is to give an example where it yields the wrong answer.}}
    \begin{enumerate}
    \item \textbf{\textcolor{blue}{Pick the activity of the shortest duration 
    from amongst those which do not overlap previously selected activities.}}
        \\ With the set of activities $A = \{a_1, a_2, a_3\}$ where their start times are $s = \{2, 0, 3\}$ and finish times are $f = \{4, 3, 6\}$. Then their respective durations are $d = \{2, 3, 3\}$.
        \\ The solution obtained would be $\{a_1\}$ whilst the optimal solution is actually $\{a_2, a_3\}$.
    \item \textbf{\textcolor{blue}{(*) Pick the activity which overlaps the fewest other remaining activities
    from amongst those which do not overlap previously selected activities.}}
        \\ With the set of activities $A = \{a_1, a_2, a_3, ..., a_9\}$ where their start times are $s = \{0, 1, 1, 2, 3, 4, 5, 5, 6\}$ and finish times are $f = \{2, 3, 3, 4, 5, 6, 7, 7, 8\}$. Then their respective number of overlaps are $L = \{2, 3, 3, 3, 2, 3, 3, 3, 2\}$
        \\ The solution obtained would be $\{a_1, a_5, a_9\}$ whilst the optimal solution is actually $\{a_1, a_9, a_4, a_6\}$.
    \item \textbf{\textcolor{blue}{Pick the activity with the earliest start time
    from amongst those which do not overlap previously selected activities.}}
        \\ With the set of activities $A = \{a_1, a_2, a_3\}$ where their start times are $s = \{0, 1, 3\}$ and finish times are $f = \{4, 2, 4\}$. 
        \\ The solution obtained would be $\{a_1\}$ whilst the optimal solution is actually $\{a_2, a_3\}$.
    \end{enumerate}
\end{enumerate}

\end{document}