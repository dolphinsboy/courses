\documentclass{article}
\usepackage[utf8]{inputenc}
\usepackage{listings}
\usepackage{color}

\title{FA lesson 4}

\begin{document}

\maketitle

\section{Master Theorem}
The master theorem \textbf{only} concerns recurrence relations of the form:
\begin{equation}T(n)=aT(\frac{n}{b}) + f(n) \textnormal{ where } a \geq 1 \textnormal{ and } b > 1\end{equation}
\newline $n$ is the size of the problem, $a$ the number of subproblems in recursion, 
$\frac{n}{b}$ the size of each subproblem and $f(n)$ the cost of the work done outside the recursive calls
such as dividing the problem or merging the solutions of the subproblems.

\subsection{Case 1: Low overhead}
\begin{equation}
\left\{
    \begin{array}{ll}
        f(n)=O(n^c) \\
        log_b\ a > c
    \end{array}
\right. \Rightarrow T(n) = \Theta(n^{log_b\ a})
\end{equation}
 
\subsection{Case 2: Perfect case?}
\begin{equation}
\left\{
    \begin{array}{ll}
        f(n)=\Theta(n^{c}log^{k}n) \textnormal{ for some constant } k \geq 0 \\
        log_b\ a = c
    \end{array}
\right. \Rightarrow T(n) = \Theta(n^{c}log^{k+1}\ n)
\end{equation}

\subsection{Case 3: High overhead}
\begin{equation}
\left\{
    \begin{array}{ll}
        f(n)=\Omega(n^c) \\
        log_b\ a < c \\
        af(\frac{n}{b}) \leq kf(n) \textnormal{ for } k < 1 \textnormal{ and sufficiently large n}
    \end{array}
\right. \Rightarrow T(n) = \Theta(f(n))
\end{equation}


\end{document}
