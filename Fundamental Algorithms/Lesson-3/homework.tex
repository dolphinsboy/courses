\documentclass{article}
\usepackage[utf8]{inputenc}
\usepackage{minted}
\usepackage{color}
\usepackage{fancyhdr}
\usepackage{lastpage}
\pagestyle{fancy}
\renewcommand{\headrulewidth}{0pt}
\cfoot[R]{\thepage~of~\pageref{LastPage}}

\title{FA homework 3}
\author{Quentin McGaw (qm301)}
\date{09 February 2017}

\begin{document}

\maketitle

\begin{quote}
Well, you see, Haresh Chacha, its like this.  First you have ten, that's
just ten, that is, ten to the first power.  Then you have a hundred, which
is ten times ten, which makes it ten to the second power.  Then you have a
thousand which is ten to the third power.  Then you have ten thousand,
which is ten to the fourth power - but this is where the problem begins,
don't you see?  We don't have a special word for that, and we really
should. \ldots  But you know, said Haresh, I think there is a special
word for ten thousand.  The Chinese tanners of Calcutta once told me that
they used the number ten-thousand as a standard unit of counting.  What
they call it I can't remember \ldots  Bhaskar was electrified.  But Haresh
Chacha you must find that number for me, he said.  You must find out what
they call it.  I have to know, he said, his eyes burning with mystical fire
and his small frog-like features taking on an astonishing radiance.
\\ -- from A Suitable Boy by Vikran Seth
\end{quote}

\begin{enumerate}
\item \textbf{\textcolor{blue}{Write each of the following functions as $\Theta(g(n))$ where
$g(n)$ is one of the standard forms:}}
    \begin{enumerate}
    \item \textbf{\textcolor{blue}{$2n^4-11n+98$}}
        \newline\newline $\Theta(g(n)) = \Theta(n^4)$
    \item \textbf{\textcolor{blue}{$6n + 43n\lg n$}}
        \newline\newline$\Theta(g(n)) = \Theta(nlg\ n)$\newline
    \item \textbf{\textcolor{blue}{$63n^2 + 14n\lg^5n$}}
        \newline\newline$\Theta(g(n)) = \Theta(n^2)$\newline
    \item \textbf{\textcolor{blue}{$3 + \frac{5}{n}$}}
        \newline\newline$\Theta(g(n)) = \Theta(1)$\newline
    \end{enumerate}

\item \textbf{\textcolor{blue}{Illustrate the operation of {\tt RADIX-SORT} on the
list: COW, DOG, SEA, RUG, ROW, MOB, BOX, TAB, BAR, EAR, TAR, DIG, BIG,
TEA, NOW, FOX following the Figure in the Radix-Sort section.
(Use alphabetical order and sort one letter at a time.)}}
    \newline\newline\textbf{Initial list:} COW, DOG, SEA, RUG, ROW, MOB, BOX, TAB, BAR, EAR, TAR, DIG, BIG, TEA, NOW, FOX
    \newline\textbf{Pass 1:} ['SEA', 'TEA', 'MOB', 'TAB', 'DOG', 'RUG', 'DIG', 'BIG', 'BAR', 'EAR', 'TAR', 'COW', 'ROW', 'NOW', 'BOX', 'FOX']
    \newline\textbf{Pass 2:} ['TAB', 'BAR', 'EAR', 'TAR', 'SEA', 'TEA', 'DIG', 'BIG', 'MOB', 'DOG', 'COW', 'ROW', 'NOW', 'BOX', 'FOX', 'RUG']
    \newline\textbf{Pass 3:} ['BAR', 'BIG', 'BOX', 'COW', 'DIG', 'DOG', 'EAR', 'FOX', 'MOB', 'NOW', 'ROW', 'RUG', 'SEA', 'TAB', 'TAR', 'TEA']
    \newline\newline The following Python code was used to generate the several passes:
    \begin{minted}{python}
    from operator import itemgetter
                 
    def radix_sort_alpha(words):
        l = len(words[0])
        for w in words:
            if len(w) != l:
                raise Exception("Words should have the same length")
        for i in range(l, 0, -1):
            words = sorted(words, key=itemgetter(i - 1))
            words_str = str([''.join(w) for w in words])
            print "PASS "+str(l - i + 1)+": "+words_str
        return words_str
            
    radix_sort_alpha(["COW", "DOG", "SEA", "RUG", "ROW", "MOB", \
                      "BOX", "TAB", "BAR", "EAR", "TAR", "DIG", \
                      "BIG", "TEA", "NOW", "FOX"])
    \end{minted}

\item \textbf{\textcolor{blue}{Illustrate the operation of {\tt BUCKET-SORT} 
(with $10$ buckets)
on the
array \\ $A=(.79,.13,.16,.64,.39,.20,.89,.53,.71,.43)$ \\  following
the Figure in the Bucket-Sort section.}}
    \newline\newline Initial input array A: 
    \newline [0.79, 0.13, 0.16, 0.64, 0.39, 0.2, 0.89, 0.53, 0.71, 0.43]
    \newline Initial output buckets array B:
    \newline [[], [], [], [], [], [], [], [], [], []]
    \newline Output buckets array B with elements in buckets:
    \newline [[], [0.13, 0.16], [0.2], [0.39], [0.43], [0.53], [0.64], [0.79, 0.71], [0.89], []]
    \newline Output buckets array B with elements sorted in buckets:
    \newline [[], [0.13, 0.16], [0.2], [0.39], [0.43], [0.53], [0.64], [0.71, 0.79], [0.89], []]
    \newline Final output array B:
    \newline [0.13, 0.16, 0.2, 0.39, 0.43, 0.53, 0.64, 0.71, 0.79, 0.89]
    \newline The following Python code was used to generate these stages:
    \begin{minted}{python}
    from math import floor
                 
    def bucket_sort(A):
        print "Initial input array A: "+str(A)
        n = len(A)
        for i in range(n):
            assert(A[i] >= 0 and A[i] < 1)
        B = [[] for _ in range(n)]
        print "Initial output buckets array B: "+str(B)
        for i in range(n):
            place = int(floor(A[i] * n))
            B[place].append(A[i])
        print "Output buckets array B with \
               elements in buckets: "+str(B)
        for j in range(n):
            B[j].sort()
        print "Output buckets array B with \
               elements sorted in buckets: "+str(B)
        B_final = []
        for bucket in B:
            B_final += bucket
        print "Final output array B: "+str(B_final)
        return B_final
        
    bucket_sort([.79,.13,.16,.64,.39,.20,.89,.53,.71,.43])
    \end{minted}

\item \textbf{\textcolor{blue}{Given $A[1\cdots N]$ with $0\leq A[I]<N^N$ for all $I$.}}
    \begin{enumerate}
    \item \textbf{\textcolor{blue}{How long will {\tt COUNTING-SORT} take?}}
        \newline For {\tt COUNTING-SORT}, $n=N$ and $k=N^N$. Hence it will take $\Theta(n+k)=\Theta(N^N)$ time.
		Note that we had to go through the array C of length $N^N$, leading to $\Theta(N^N)$
    \item \textbf{\textcolor{blue}{How long will {\tt RADIX-SORT} take using base $N$?}}
        \newline For {\tt RADIX-SORT}, $n=N$ and $k=N^N$ and $b=N$.
		Hence it will take $\Theta((n+b)log_b k)=\Theta((N+N)log_N N^N)=\Theta((N+N)N)=\Theta(N^2)$ time.
		Note that the array C is of lenght $N$ and that each element has $N$ digits, so this leads to $\Theta(N^2)$
    \item \textbf{\textcolor{blue}{How long will {\tt RADIX-SORT} take using base $N^{\sqrt{N}}$?
    (Assume $\sqrt{N}$ integral.)}}
        \newline For {\tt RADIX-SORT}, $n=N$, $k=N^N$ and $b=N^\sqrt{N}$.
		Hence it will take $O((n+b)log_b k)=O((N+N^{\sqrt{N}}) log_{N^\sqrt{N}}\ N^N)=O((N+N^{\sqrt{N}})2=O(N^{\sqrt{N}})$ time.
		Note that there are $\sqrt{N}$ digits per element, and the length of the array C is $N^{\sqrt{N}}, leading to \Theta(\sqrt(N)N^{\sqrt{N}}).
		XXX
	\item \textbf{\textcolor{blue}{(*) Argue that for no base $K$ will {\tt RADIX-SORT} do as well
	as {\tt HEAP-SORT}.}}
		\newline If the base $K$ has $K\geq N^2$ then it takes time
		$N^2$ or more just to go through the array $C$, which is 
		worse than the $\Theta(N\lg N)$ of {\tt HEAP-SORT}.
		On the other side, if the base $K$ has $K\leq  N^2$ then 
		the number of digits is at least $N/2$ (the smaller the base, 
		the greater the number of digits, and with base $N^2$, there 
		are $N/2$ digits) and as each digit takes linear time the
		total time is $\Omega(N^2)$ which is worse than {\tt HEAP-SORT}.
		XXX
    \end{enumerate}
    
\item \textbf{\textcolor{blue}{Write the time $T(N)$ (don't worry about the output!) for the following algorithms in the
form $T(N)=\Theta(g(N))$ for a standard $g(N)$.  
For time, consider the
total number of times X++, I=2*I, J++,J=2*J respectively are applied.
(Note: * means multiplication, ++ means increment one.)  The hardest is the last one,
there is an outer FOR I loop, write the time it takes inside the loop as a function of $I$ and $N$.
Then try (!) to add over $I=1$ to $N$.}}
    \begin{enumerate}
    \item \textbf{\textcolor{blue}{
    X=0
    \\ FOR I=1 TO N
    \\ \hspace*{1cm} do FOR J=1 TO N
    \\ \hspace*{2cm} X ++ }}
        \newline\newline $T(N)=\Theta(N^2)$ because it is a double loop.
    \item \textbf{\textcolor{blue}{
    I=1
    \\ WHILE I $<$ N
    \\ \hspace*{1cm} do I = 2*I}}
        \newline\newline $T(N)=\Theta(log_2\N)=\Theta(log\ N)$ because we 
		only need $lg\N$ doublings to get to $N$.
    \item \textbf{\textcolor{blue}{
    FOR I=1 TO N
    \\ \hspace*{1cm} do J=1
    \\ \hspace*{1cm} WHILE J*J $<$ I
    \\ \hspace*{2cm} do J ++}}
        \newline\newline 
		For each $I$ the subloop takes $O(\sqrt{I})$ as after that $J*J>I$.
		So we need look at $\sqrt{1}+...+\sqrt{N}$. 
		This is at most $N\sqrt{N}$ as there are $N$ terms and each is 
		at most $\sqrt{N}$.  As a lower bound cut it off at the middle and 
		we have $\sqrt{N/2} + ... + \sqrt{N}$. There are $N/2$ terms 
		here and each is at least $\sqrt{N/2}$ so the total is at 
		least $N\sqrt{N}/(2\sqrt{2})$. Thus $T(N)=\Theta(N^{3/2})$.
    \item \textbf{\textcolor{blue}{
    FOR I = 1 to N
    \\ \hspace*{1cm} J=I
    \\ \hspace*{1cm} WHILE J $<$ N
    \\ \hspace*{2cm} do J=2*J}}
        \newline\newline For a given $I$ the subloop starts at $I$ 
		and double until reaching $N$.  We double $t$ times, where $t$ 
		is the smallest integer $I2^t\geq N$, so 
		$t=\lceil \lg(N/I) \rceil$. So the total time is
		\begin{equation}TOTAL = O\left( \sum_{i=1}^n \lceil\lg(n/i)\rceil \right)\end{equation}
		This is challenging.  
		\newline Approach One: We get ridof the ceiling by noting that 
		the ceiling can only affect each term by $1$ and therefore 
		the sum by at most $n$ and so 
		\begin{equation}TOTAL = O(n)+O\left( \sum_{i=1}^n \lg(n/i) \right)\end{equation}
		Now there are a variety of approaches. One is via Stirling's 
		Formula. The object in parentheses is precisely $\ln(n^n/n!)$ 
		and by Stirling $n^n/n! \sim e^n(2\pi n)^{-1/2}$.
		The square root term is inconsequential and 
		$\lg(n^n/e^n) \sim n\lg 2 = O(n)$. Thus 
		\begin{equation}TOTAL = O(n) + O(n ) = O(n)\end{equation}
		This is a linear time algorithm!  Note that the ceiling 
		actually does have an effect on the constants buried in 
		the $O$ as both parts are linear in $n$.
		{\tt Comment:} How did we know that removing the ceilings 
		would work? We didn't, we tried it and it turned out its 
		effect was not dominant so we were OK. This is part of 
		the {\em art} of doing asymptotics!
		\newline Approach Two: Similar to the analysis of 
		BUILD-MAX-HEAP we have $1$ doubling $n/2$ times, 
		$2$ doublings $n/4$ times ($n/4 < i \leq n/2$), 
		$3$ doublings $n/8$ times so the total doublings is 
		$n \sum_u u2^{-u}$ but that sum, even to infinity, is $2$
		so the total doublings is $\sim 2n$.
    \end{enumerate}
\item \textbf{\textcolor{blue}{Prof.\ Squander decides to do Bucket Sort on $n$ items with $n^2$ buckets while
his student Ima Hogg decides to do Bucket Sort on $n$ items with $n^{1/2}$ buckets.
Assume that the items are indeed uniformly distributed.  Assume that Ima's algorithm
for sorting inside a bucket takes time $O(m^2)$ when the bucket has $m$ items.}}
    \begin{enumerate}
    \item \textbf{\textcolor{blue}{Argue that Prof.\ Squander has made a poor choice of the number of buckets by
    looking analyzing the time of Bucket Sort in his case.}}
        \newline\newline For Prof. Squander, we have $n=n$ and $k=n^2$ so we require $\Theta(n + n^2)=\Theta(n^2)$ time, which is the worst case scenario for a bucket sort (upper bond).
    \item \textbf{\textcolor{blue}{Argue that Ima has made a poor choice of the number of buckets by looking analyzing the time of Bucket Sort in her case.}}
        \newline\newline For Ima Hogg's bucket sort, because of the uniform distribution, there should be $n^\frac{1}{2}$ items per bucket so sorting each bucket takes $\Theta((n^\frac{1}{2})^2)=\Theta(n)$ thus we require $\Theta(n^\frac{3}{2})$ which is still worst than $\Theta(n + k)$.
    \item \textbf{\textcolor{blue}{Argue that Ima uses roughly the same amount of {\em space} as someone
    using $n$ buckets.}}
        \newline\newline Because there is $n^\frac{1}{2}$ items per bucket and there are $n^{\frac{1}{2}}$ buckets, the total space used is $n^\frac{1}{2}*n^\frac{1}{2}=n$ which is obviously the same as the space used by someone using $n$ buckets.
    \end{enumerate}
\end{enumerate}

\begin{quote}
Every universe, our own included, begins in conversation.
\\ -- Michael Chabon
\end{quote}

\end{document}
