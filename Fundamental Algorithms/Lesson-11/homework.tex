\documentclass[11pt]{article}
\pagestyle{empty}
\usepackage{color}
\usepackage{amsfonts}
\usepackage{amsmath}
\usepackage{fancyhdr}
\usepackage{lastpage}
\usepackage{enumitem}
\usepackage[options]{algorithm2e}
\pagestyle{fancy}
\renewcommand{\headrulewidth}{0pt}
\cfoot[R]{\thepage~of~\pageref{LastPage}}
\addtolength{\oddsidemargin}{-.875in}
\addtolength{\evensidemargin}{-.875in}
\addtolength{\textwidth}{1.75in}
\addtolength{\topmargin}{-.875in}
\addtolength{\textheight}{1.75in}

\begin{document}
\begin{center} {\Large\bf FA, Homework 11} \\ Quentin McGaw (qm301) \\ 04/27/17
\end{center}

\begin{quote}
My work has always tried to unite the true with the beautiful and
when I had to choose one or the other I usually chose the beautiful.
--  Hermann Weyl
\end{quote}

\begin{algorithm}[H]
\SetKwFunction{mstprim}{MST-PRIM}
\Indm\mstprim{G, w, r} \\
\Indp
    \For{each u $\in$ G.V}{
        key[u] = $\infty$ \\
        $\pi$[u] = NIL \\
    }
    key[r] = 0 \\
    Q = G.V \\
    S = [] \\
    \While{Q $\neq \emptyset$}{
        u = EXTRACT-MIN(Q) \\
        \For{each v $\in$ G.Adj[u]}{
            \If{v $\in$ Q and key[v] $>$ w(u,v)}{
                $\pi$[v] = u \\
                key[v] = w(u,v) \\
            }
        }
        S.add(u) \\
    }
\caption{Prim's algorithm}
\end{algorithm}

\begin{enumerate}
\item \textbf{\textcolor{blue}{Consider {\tt Dumb Prim} for MST.  The high level idea is the same but to find the minimal weight of an edge $\{v,w\}$, $v\in S$, $w\not\in S$, one looks at all the weights and finds the minimum in the usual way. (There is no updating in {\tt Dumb Prim}.) Assume that all pairs $\{v,w\}$ have a weight.  Let $n$ be the number of vertices.}}
\begin{enumerate}
    \item \textbf{\textcolor{blue}{When $|S|=i$ what is the time to add a vertex to $S$ as a function of $n$ and $i$.}}
        \\ Let $\overline{S}$ be the complement of $S$. We assume we use linked lists to store $S$ and $\overline{S}$. To add a vertex from $\overline{S}$ to $S$, we need to:
        \begin{itemize}
            \item Add it to $S$, requiring O(1) time.
            \item Remove it from $\overline{S}$, requiring O(n) time.
            \item Search in a double for loop for the minimum, requiring $O(i(n-i))$ time.
        \end{itemize}
        Overall, this thus takes $O(i(n-i))$ time.    
    \item \textbf{\textcolor{blue}{What is the total time for {\tt Dumb Prim} as a sum over $i$.}}
        \\ Total time is $O(\sum_{i=1}^{n-1} i(n-i))$ because we start with $|S|$ = 1 and finish when $|S|$ = n.
    \item \textbf{\textcolor{blue}{Evaluate the above sum as $\Theta(g(n))$ for some nice function $g(n)$. (Caution:  The time is {\em not} an increasing function of $i$.  For example, when $i=n-1$ the time is quite quick.)}}
        \begin{itemize}
            \item There are $n-1$ terms
            \item The \textbf{biggest} term is the middle when $i = \frac{n}{2}$ with $i(n-i) = \frac{n^2}{4}$.
            \item The sum is at most $\frac{(n-1)n^2}{4} \approx \frac{n^3}{4}$
            \item At $i=\frac{n}{4}$, $i(n-i) = \frac{3n^2}{16}$ and all terms from $i=\frac{n}{4}$ to $i=\frac{3n}{4}$ are at least that big.
            \item The sum is at least $(\frac{n}{2})(\frac{3n^2}{16}) \approx \frac{n^3}{10}$
            \item As consequence, the sum is $O(n^3)$
        \end{itemize}
    \item \textbf{\textcolor{blue}{Compare the time for {\tt Dumb Prim} with {\tt Prim} as discussed in class}}
        \\ Prim takes $O(E\ln(V))$ where:
        \begin{itemize}
            \item $V = n$
            \item $E = \left(\begin{smallmatrix} n \\ 2 \end{smallmatrix}\right) \approx \frac{n^2}{2}$
        \end{itemize}
        Prim therefore requires $O(n^2\ln(n))$, whereas Dumb Prim requires $O(n^3)$.
\end{enumerate}

\item \textbf{\textcolor{blue}{Consider Prim's Algorithm for MST on the complete graph with vertex set $\{1,\ldots,n\}$.  Assume that edge $\{i,j\}$ has weight $(j-i)^2$.  Let the root vertex $r=1$.  Show the pattern as Prim's Algorithm is applied. In particular, Let $n=100$ and consider the situation when the tree created has $73$ elements and $\pi$ and $key$ have been updated.}}
    \\ Because it is a complete graph, the lightest edges will be edges of weight 1 between consecutive vertices numbers. 
    The set S grows from $\{1\}$ to $\{1,2\}$ to $\{1,2,3\}$ to ... to $\{1,2,3, ... n\}$
    When S = $\{1,2,...,i\}$, the closest point will be $i+1$ and $\pi$[$i+1$] = $i$ with key[$i+1$] = 1.
\begin{enumerate}
    \item \textbf{\textcolor{blue}{What are these $73$ elements.}}
        \\ 1, 2, 3 ... 73.
    \item \textbf{\textcolor{blue}{What are $\pi[84]$ and $key[84]$.}}
        \\ As we have 73 elements in our MST so far, then for the vertex 84:
        \begin{itemize}
            \item $\pi[84]$ = 73
            \item key[84] = $(84 - 73)^2 = 121$
        \end{itemize}
\end{enumerate}

\item \textbf{\textcolor{blue}{Find $d=\gcd(89,55)$ and $x,y$ with $89x+55y=1$. [Remark: This is part of a pattern with two consecutive numbers from the Fibonacci sequence $0,1,1,2,3,5,8,13,21,34,55,89,\ldots$]}}
    \begin{enumerate}
        \item First method (Fibonacci)
        \begin{itemize}
            \item We prove that any two consecutive terms  of the Fibonacci sequence are relatively prime with the following:
            \\ We have $f_1 = 1$, $f_2 = 1$ so $\gcd(f_1, f_2) = 1$.
            \\ Suppose $\gcd(f_n, f_{n+1}) = 1$. We will prove that $\gcd(f_{n+1}, f_{n+2}) = 1$.
            \\ Because $f_{n+2} = f_{n+1} + f_n$, we have $\gcd(f_{n+1}, f_{n+2}) = \gcd(f_{n+1}, f_{n+1} + f_n)$.
            \\ Using the gcd property, we obtain $\gcd(f_{n+1}, f_{n+1} + f_n) = \gcd(f_{n+1}, f_n) = 1$.
            \\ Hence $\gcd(f_n, f_{n+1}) = 1$ for all $n > 0$.
            \item Because 55 and 89 are two consecutive numbers from the Fibonacci sequence, they are hence relatively prime and their gcd $d$ is $1$.
        \end{itemize}
        \item Second method (Euclidian algorithm)
        \begin{itemize}
            \item $89 = 55 \times 1 + 34$
            \item $55 = 34 \times 1 + 21$
            \item $34 = 21 \times 1 + 13$
            \item $21 = 13 \times 1 + 8$
            \item $13 = 8 \times 1 + 5$
            \item $8 = 5 \times 1 + 3$
            \item $5 = 3 \times 1 + 2$
            \item $3 = 2 \times 1 + 1$
            \item $2 = 1 \times 1 + 1$
            \item $1 = \textcolor{red}{1} \times 1 + 0$ so the gcd is 1. Therefore $d = \gcd(89, 55) = 1$
        \end{itemize}
    \end{enumerate}
    Now to solve $89x+55y=1$, we first need to see that $89x+55y=1 \Rightarrow 89x+55y=\gcd(89, 55)$.
    Because $\gcd(89, 55) = 1$, it divides $1$ and thus $89x + 55y$ so there is an infinite set of solutions $x, y$.
    \begin{itemize}
        \item $1 = 3 - 1(2)$
        \item $1 = 3 - 1(5-3)$
        \item $1 = 2(3) - 1(5)$
        \item $1 = 2(8-5) - 1(5)$
        \item $1 = 2(8) - 3(5)$
        \item $1 = 2(8) - 3(13-8)$
        \item $1 = 5(8) - 3(13)$
        \item $1 = 5(21-13) - 3(13)$
        \item $1 = 5(21) - 8(13)$
        \item $1 = 5(21) - 8(34-21)$
        \item $1 = 13(21) - 8(34)$
        \item $1 = 13(55-34) - 8(34)$
        \item $1 = 13(55) - 21(34)$
        \item $1 = 13(55) - 21(89-55)$
        \item $1 = 34(55) - 21(89)$
    \end{itemize}
     Now we can see that $1 = 89 \times (-21) + 55 \times 34 $ therefore one solution is $(x, y) = (-21, 34)$.
     The general solutions have the form:
     \\ $x = -21 + \frac{55r}{1} = -21 + 55r$
     \\ $y = 34 - \frac{89r}{1} = 34 - 89r$
     \\ with $r \ \epsilon\ \mathbb{Z}$
     \\ \textbf{A faster way to find the GCD: } EUCLID(89,55) = EUCLID(55,89-55) = EUCLID(34,21) = EUCLID(21,13) = EUCLID(13,8) = EUCLID(8,5) = EUCLID(5,3) = EUCLID(3,2) = EUCLID(2,1) = EUCLID(\textcolor{green}{1},0) = 1
     


\item \textbf{\textcolor{blue}{Find $\frac{211}{507}$ in $Z_{1000}$.}}
    \begin{itemize}
        \item We want to find $x$ such that $507x = 1$ in $Z_{1000}$.
            \\ We use the extended-euclid algorithm to find $x$ with EUCLID(1000,507). \\
            \begin{algorithm}[H]
            \SetKwFunction{eeuclid}{EXTENDED-EUCLID}
            \Indm\eeuclid{a, b}\\
            \Indp
                \eIf{b == 0}{
                    \Return{(a, 1, 0)} \\
                }{
                    (d, x', y') = \eeuclid{b, a mod b} \\
                    x = y' \\
                    y = x' - ${\lfloor}\frac{a}{b}{\rfloor}$y' \\
                    \Return{(d, x, y)} \\
                }
            \caption{EXTENDED-EUCLID algorithm}
            \end{algorithm}
            We start by filling the columns a, b and ${\lfloor}\frac{a}{b}{\rfloor}$ until b = 0.
            \\ d is just the gcd so we can fill the columns with 1s once it is found. \\
            \begin{center}
            \begin{tabular}{| c | c | c | c | c | c |} 
            \hline
            a & b & ${\lfloor}\frac{a}{b}{\rfloor}$ & d & x & y \\
            \hline
            1000 & 507 & 1 & 1 & 0 & 0 \\
            507 & 493 & 1 & 1 & 0 & 0 \\
            493 & 14 & 35 & 1 & 0 & 0 \\
            14 & 3 & 4 & 1 & 0 & 0 \\
            3 & 2 & 1 & 1 & 0 & 0 \\
            2 & 1 & 2 & 1 & 0 & 0 \\
            1 & 0 & - & 1 & 0 & 0 \\
            \hline
            \end{tabular}
            \end{center}
            We know that when b = 0, d = a = 1, x = 1 and y = 0 from the algorithm. \\
            \begin{center}
            \begin{tabular}{| c | c | c | c | c | c |} 
            \hline
            a & b & ${\lfloor}\frac{a}{b}{\rfloor}$ & d & x & y \\
            \hline
            1000 & 507 & 1 & 1 & 0 & 0 \\
            507 & 493 & 1 & 1 & 0 & 0 \\
            493 & 14 & 35 & 1 & 0 & 0 \\
            14 & 3 & 4 & 1 & 0 & 0 \\
            3 & 2 & 1 & 1 & 0 & 0 \\
            2 & 1 & 2 & 1 & 0 & 0 \\
            1 & 0 & - & 1 & 1 & 0 \\
            \hline
            \end{tabular}
            \end{center}
            Then starting from the bottom: \\
            \begin{itemize}
                \item We propagate y values from bottom right to top left (x = y')
                \item We find y = x' - ${\lfloor}\frac{a}{b}{\rfloor}$y'
                \item Note that x' and y' are values of the row below the current row.
            \end{itemize}
            \begin{center}
            \begin{tabular}{| c | c | c | c | c | c |} 
            \hline
            a & b & ${\lfloor}\frac{a}{b}{\rfloor}$ & d & x & y \\
            \hline
            1000 & 507 & 1 & 1 & 181 & -176 - 181 $\times$ 1 \\
            507 & 493 & 1 & 1 & -176 & 5 + 176 $\times$ 1 \\
            493 & 14 & 35 & 1 & 5 & -1 - 5 $\times$ 35 \\
            14 & 3 & 4 & 1 & -1 & 1 + 1 $\times$ 4 \\
            3 & 2 & 1 & 1 & 1 & 0 - 1 $\times$ 1 \\
            2 & 1 & 2 & 1 & 0 & 1 - 1 $\times$ 0 \\
            1 & 0 & - & 1 & 1 & 0 \\
            \hline
            \end{tabular}
            \end{center}
            So we finally obtain: \\
            \begin{center}
            \begin{tabular}{| c | c | c | c | c | c |} 
            \hline
            a & b & ${\lfloor}\frac{a}{b}{\rfloor}$ & d & x & y \\
            \hline
            1000 & 507 & 1 & 1 & 181 & -357 \\
            507 & 493 & 1 & 1 & -176 & 181 \\
            493 & 14 & 35 & 1 & 5 & -176 \\
            14 & 3 & 4 & 1 & -1 & 5 \\
            3 & 2 & 1 & 1 & 1 & -1 \\
            2 & 1 & 2 & 1 & 0 & 1 \\
            1 & 0 & - & 1 & 1 & 0 \\
            \hline
            \end{tabular}
            \end{center}
            Now, we have 181 $\times$ 1000 + (-357) $\times$ 507 = gcd(507, 1000) = 1.
            \\ As a consequence, (-357)(507) = 1 in $Z_{1000}$.
        \item In $Z_{1000}$, $(-357)(507) = 1 \Rightarrow \frac{1}{507} = -357 = 643$
        \item As a consequence, $\frac{211}{507} = (211)(643) = 135673 = 673$
        \item We can check that $673 \times 507 = 341211 = 211$ in $Z_{1000}$.
    \end{itemize}

    

\item \textbf{\textcolor{blue}{Solve the system \\ $x\equiv 34 \bmod{101}$\\ $x\equiv 59 \bmod{103}$.}}
    \begin{itemize}
        \item The second equation gives $x \equiv 59 \bmod{103} \Rightarrow x = 103q + 59$, for $q\ \epsilon\ \mathbb{Z}$.
        \item We substitute this $x$ into the first equation, giving: $103q + 59 \equiv 34 \bmod{101} \Rightarrow 103q \equiv 76 \bmod{101} \Rightarrow 2q \equiv 76 \bmod{101}$.
        \item Because 2 and 101 are relatively prime, we obtain $q \equiv 38 \bmod{101}$
        \item Hence the solution is $x \equiv 103 \times 38 + 59 \bmod{101 \times 103} \equiv 3973 \bmod{10403}$
    \end{itemize}

\item \textbf{\textcolor{blue}{Using the Island-Hopping Method to find $2^{1000}$ modulo $1001$ using a Calculator but NOT using multiple precision arithmetic. (You should never have an intermediate value more than a million.)}}
    \begin{itemize}
        \item We decompose 1000 in powers of 2, giving 1000 = 512 + 256 + 128 + 64 + 32 + 8.
        \item So we have $2^{1000} = 2^{512} \times 2^{256} \times 2^{128} \times 2^{64} \times 2^{32} \times 2^8$.
        \item Now we have to figure each of these term with modulo 1001 (we can start at $2^8$ directly):
        \begin{itemize}[label={***}]
            \item $2^{8} = 256 \bmod{1001}$
            \item $2^{16} = 256 \times 256 \bmod{1001} = 471$
            \item $2^{32} = 471 \times 471 \bmod{1001} = 620$
            \item $2^{64} = 620 \times 620 \bmod{1001} = 16$
            \item $2^{128} = 16 \times 16 \bmod{1001} = 256$
            \item $2^{256} = 256 \times 256 \bmod{1001} = 471$
            \item $2^{512} = 471 \times 471 \bmod{1001} = 620$
        \end{itemize}
        \item Therefore $2^{1000} \bmod{1001} = 620 \times 471 \times 256 \times 16 \times 620 \times 256 \bmod{10001}$
        \item Now we need to process multiplications at each stage and with mod 1001:
        \begin{itemize}[label={***}]
            \item $620 \times 471 \bmod{1001} = 729$
            \item $729 \times 256 \bmod{1001} = 438$
            \item $438 \times 16 \bmod{1001} = 1$
            \item $1 \times 620 \bmod{1001} = 620$
            \item $620 \times 256 \bmod{1001} = 562$          
        \end{itemize}
        Therefore $2^{1000} \bmod{1001} = 562$.
    \end{itemize}

\end{enumerate}

\begin{quote}
The universe is not only queerer than we suppose but queerer than we {\em can}
suppose.  \\ -- J.B.S. Haldane
\end{quote}

\end{document}