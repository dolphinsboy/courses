\documentclass[11pt]{article}
\pagestyle{empty}
\usepackage{color}
\usepackage{minted}
\usepackage{fancyhdr}
\usepackage{lastpage}
\usepackage[options]{algorithm2e}
\pagestyle{fancy}
\renewcommand{\headrulewidth}{0pt}
\cfoot[R]{\thepage~of~\pageref{LastPage}}
\addtolength{\oddsidemargin}{-.875in}
\addtolength{\evensidemargin}{-.875in}
\addtolength{\textwidth}{1.75in}
\addtolength{\topmargin}{-.875in}
\addtolength{\textheight}{1.75in}
\begin{document}
\begin{center} {\Large\bf FA, Homework 2}  \\ Quentin McGaw (qm301) \\ 02/09/17
\end{center}

\textbf{Quick-sort} \\\\
\begin{tabular}{c|c|c}
    Worst & Average & Best \\
    \hline
    $O(n^2)$ & $O(n\log(n))$ & $O(n\log(n))$
\end{tabular}
\\
\begin{minted}{python}
class QuickSort(object):
    def sort(self, A):
        self.quickSortHelper(A, 0, len(A)-1)

    def quickSortHelper(self, A, p, r):
        if p < r:
            q = self.partition(A, p, r)
            self.quickSortHelper(A, p, q-1)
            self.quickSortHelper(A, q+1, r)

    def partition(self, A, p, r):
        pivot = A[r]
        i = p - 1
        for j in range(p, r+1):
            if A[j] <= pivot:
                i += 1
                A[i], A[j] = A[j], A[i]
        return i
\end{minted}

\textbf{Counting-sort} \\\\
\begin{tabular}{c|c|c}
    Worst & Average & Best \\
    \hline
    $O(n+k)$ & $O(n+k)$ & $O(n+k)$
\end{tabular}
\\
\begin{minted}{python}
class CountingSort(object):
    def sort(self, A, B, k=None):
        """
            Args:
                A (list): Unordered array
                B (list): Output sorted array
                k (int): Range of values = Number of buckets - 1
        """
        if k is None:
            k = max(A)
        n = len(A)
        C = [0 for _ in range(k+1)] #counter, k is the range of values
        for i in range(n):
            C[A[i]] += 1
        # C[i] contains the number of elements = i
        for i in range(1, k+1):
            C[i] += C[i-1] 
        # C[i] contains the number of elements <= i
        for i in range(n-1, -1, -1): #n to 0
            place = C[A[i]]
            B[place - 1] = A[i] # -1 for Python
            C[A[i]] -= 1
\end{minted}

\textbf{Radix-sort} \\\\
\begin{tabular}{c|c|c}
    Worst & Average & Best \\
    \hline
    $O(nk)$ & $O(nk)$ & $O(nk)$
\end{tabular}
\begin{itemize}
    \item $k$ is the base with $D$ digits ($0 \leq A[i] < k^D$)
    \item input A[1..n] all$0 <= A[i] < k^D$ integers written in base K with D digits
    \item for j=D to 1, apply sort to jth digit
\end{itemize}
\begin{minted}{python}
class RadixSort(object):
    def sort(self, A, RADIX=10):
        placement = 1
        tmp = A[0]
        while tmp > 0:
            buckets = [[] for _ in range(RADIX)]
            for i in range(len(A)):
                tmp = A[i] / placement
                buckets[tmp % RADIX].append(A[i])
            i = 0
            for bucket in buckets:
                for x in bucket:
                    A[i] = x
                    i += 1
            placement *= RADIX # move to next digit
\end{minted}


\begin{enumerate}

\item  \textbf{\textcolor{blue}{Illustrate the operation of PARTITION(A,1,12) on the array 
\[ A=(13,19,9,5,12,8,7,4,11,2,6,10) \] 
(You may use either the text's program or the version given in class, but please specify which you are using.)}}
    \\ Note that PARTITION is used in QUICK-SORT (see \textbf{above} and \textbf{below})
    \begin{algorithm}
        \SetKwFunction{partition}{PARTITION}
        \Indm\partition{A, p, r} \\
        \Indp
            pivot = A[r] \\
            i = p - 1 \\
            \For{j from p to r}{
                \If{A[j] $\leq$ pivot}{
                    i$++$ \\
                    A[i], A[j] = A[j], A[i] \\
                }
            }
            \Return{i}
        \caption{PARTITION}
    \end{algorithm}
    \begin{algorithm}
        \SetKwFunction{quicksort}{QUICK-SORT}
        \Indm\quicksort{A, p, r} \\
        \Indp
            \If{p $<$ r}{
                q = \partition{A, p, r} \\
                \quicksort{A, p, q-1} \\
                \quicksort{A, q+1, r} \\
            }
        \caption{QUICK-SORT}
    \end{algorithm}
    We start with A = (13, 19, 9, 5, 12, 8, 7, 4, 11, 2, 6, 10) and p = 0, r = 11.
    \begin{itemize}
        \item pivot = A[r] = 10 (last element)
        \item i = p - 1 = -1
        \item In the for loop, we start at j = p = 0 (up to r = 11).
        \item j = 0: A[j] = 13 $>$ pivot
        \item j = 1: A[j] = 19 $>$ pivot
        \item j = 2: A[j] = 9 $\leq$ pivot, so i = 0; A[0] and A[2] exchanged and A = (9,19,13,5,12,8,7,4,11,2,6,10)
        \item j = 3: A[j] = 5 $\leq$ pivot, so i = 1; A[1] and A[3] exchanged and A = (9,5,13,19,12,8,7,4,11,2,6,10)
        \item j = 4: A[j] = 12 $>$ pivot
        \item j = 5: A[j] = 8 $\leq$ pivot, so i = 2; A[2] and A[5] exchanged and A = (9,5,8,19,12,13,7,4,11,2,6,10)
        \item j = 6: A[j] = 7 $\leq$ pivot, so i = 3; A[3] and A[6] exchanged and A = (9,5,8,7,12,13,19,4,11,2,6,10)
        \item j = 7: A[j] = 4 $\leq$ pivot, so i = 4; A[4] and A[7] exchanged and A = (9,5,8,7,4,13,19,12,11,2,6,10)
        \item j = 8: A[j] = 11 $>$ pivot
        \item j = 9: A[j] = 2 $\leq$ pivot, so i = 5; A[5] and A[9] exchanged and A = (9,5,8,7,4,2,19,12,11,13,6,10)
        \item j = 10: A[j] = 6 $\leq$ pivot, so i = 6; A[6] and A[10] exchanged and A = (9,5,8,7,4,2,6,12,11,13,19,10)
        \item j = 11: A[j] = 10 $\leq$ pivot, so i = 7; A[7] and A[11] exchanged and A = (9,5,8,7,4,2,6,10,11,13,19,12)
        \item i = 7 is then returned.
    \end{itemize}

\item \textbf{\textcolor{blue}{Let $L(n)$, (``L'' for lucky) denote the number of comparisons that quicksort does if each time it is applied the pivot lies in the precise center of the array. For example, applying quicksort to an array of length $31$, say $A(1)\cdots A(31)$ objects, there would be $30$ comparisons (between $A(31)$ and all the other $A(j)$) and then $A(31)$ would end up in the $16^{th}$ place and there would be two recursive calls to quicksort on arrays each of size $15$. Find the {\em precise} value of $L(1023)$. (Hint: thats one less than 1024!)}}
    \\ Let the number of comparisons be $L(n) = p(n) + L(l) + L(r)$, where $p(n)$ is the comparisons in PARTITION.
    \begin{itemize}
        \item $p(n) = n - 1$ comparisons
        \item $L(1) = p(1) = 0$
        \item If $n$ is even, $\left\{\begin{array}{ll}
                                l = \lceil \frac{n-1}{2} \rceil \\
                                r = \lfloor \frac{n-1}{2} \rfloor
                                \end{array}\right.$
        \item If $n$ is odd, $l = r = \frac{n-1}{2}$
        \item $n = 1023$ is odd so $l = r = \frac{n-1}{2}$
        \item Write lines from $L(1023)$, $L(511)$, ... $L(1)$ and then \textbf{calculate} starting from $L(1)$
            \\
            \begin{tabular}{rclcrclcrcl}
            \\ L(1023) &=& 1022 &+& 2L(511) &=&1022 &+& 2*3586 &=& 8194
            \\ L(511)  &=&  510 &+& 2L(255) &=& 510 &+& 2*1538 &=& 3586
            \\ L(255)  &=&  254 &+& 2L(127) &=& 254 &+& 2*642  &=& 1538
            \\ L(127)  &=&  126 &+& 2L(63)  &=& 126 &+& 2*258  &=& 642
            \\ L(63)   &=&   62 &+& 2L(31)  &=&  62 &+& 2*98   &=& 258
            \\ L(31)   &=&   30 &+& 2L(15)  &=&  30 &+& 2*34   &=& 98
            \\ L(15)   &=&   14 &+& 2L(7)   &=&  14 &+& 2*10   &=& 34
            \\ L(7)    &=&    6 &+& 2L(3)   &=&   6 &+& 2*2    &=& 10
            \\ L(3)    &=&    2 &+& 2L(1)   &=&   2 &+& 2*0    &=& 0
            \\ L(1)    &=&    0  
            \end{tabular}
    \end{itemize}

\item \textbf{\textcolor{blue}{You wish to sort five elements, denoted $a,b,c,d,e$. {\em Assume} that you already know that $a<b$, $c<d$ and $a<c$. Sort the elements with 4 further comparisons. (This actually gives a sorting of $a,b,c,d,e$ under no assumptions with $7$ comparisons. For if you begin by comparing $a,b$ and then comparing $c,d$ and then comparing the smaller of $a,b$ to the smaller of $c,d$ you will have something like $a<b$, $c<d$, $a<c$ except maybe with the letter interchanged. So the $4$ more comparisons will suffice.}}
    \\ We know:
    \begin{itemize}
        \item $a < b$
        \item $c < d$
        \item $a < c$
    \end{itemize}
    From this, we have $a < c < d$.
    \begin{itemize}
        \item We compare $e$ with $c$
        \item We then compare $e$ with either $a$ or $d$
    \end{itemize}
    $\{a, c, d, e\}$ are now ranked.
    \begin{itemize}
        \item We know $b > a$ so we only need to find its rank compare to $c$, $d$ and $e$.
        \item We compare $b$ with the middle of $\{c,d,e\}$
        \item We then compare $b$ with the lower value or higher value, depending on the previous.
    \end{itemize}
    $\{a, b, c, d, e\}$ are now ranked, and we used 4 comparisons overall.

\item \textbf{\textcolor{blue}{Babu is trying to sort $a,b,c,d,e$ with seven comparisons.  First
he asks ``Is $a  < b$" and the answer is yes.  Now he asks ``Is $a < c?$"
Argue that (in worst-case) he will not succeed.}}
    \begin{itemize}
        \item There are $5$ elements to be compared so $m = 5! = 120$ permutations initially.
        \item If the answer is Yes (worst case), there are only $m = 40$ possible permutations.
        \item We are left with $7 - 2 = 5$ comparisons
        \item \textbf{Decision tree lower bound}: We don't have $40 \leq 2^5$ so we need 8 comparisons overall.
    \end{itemize}
    
\item \textbf{\textcolor{blue}{Illustrate the operation of {\tt COUNTING-SORT} with $k=6$ on
the array $A=(6,0,2,2,0,1,3,4,6,1,3)$.}}
    \begin{algorithm}
        \SetKwFunction{countingsort}{COUNTING-SORT}
        \Indm\countingsort{A, B, k} \\
        \Indp
            n = A.length \\
            C = [0..n] \\
            \For{i from 0 to n}{
                C[i] = 0 \\
            }
            \For{i from 1 to n}{
                C[A[i]]$++$ \\
            }
            \tcp{C[i] contains the number of elements = i} \\
            \For{i from 1 to k}{
                C[i] += C[i-1] \\
            }
            \tcp{C[i] contains the number of elements <= i} \\
            \For{i from n to 1}{
                place = C[A[i]] \\
                B[place] = A[i] \\
                C[A[i]]$--$ \\
            }          
        \caption{PARTITION, k should be max(A) and B is the output}
    \end{algorithm}
    \begin{itemize}
        \item C = [0, 0, 0, 0, 0, 0, 0]
        \item C = [2, 2, 2, 2, 1, 0, 2]
        \item C = [2, 4, 6, 8, 9, 9, 11]
        \item i = 11, place = C[3] = 8, so B[8] = 3 and C[3] = 7
        \item i = 10, place = C[1] = 4, so B[4] = 1 and C[1] = 3
        \item i = 9, place = C[6] = 11, so B[11] = 6 and C[6] = 10
        \item i = 8, place = C[4] = 9, so B[9] = 4 and C[4] = 8
        \item i = 7, place = C[3] = 7, so B[7] = 3 and C[3] = 6
        \item i = 6, place = C[1] = 3, so B[3] = 1 and C[1] = 2
        \item i = 5, place = C[0] = 2, so B[2] = 0 and C[0] = 1
        \item i = 4, place = C[2] = 6, so B[6] = 2 and C[2] = 5
        \item i = 3, place = C[2] = 5, so B[5] = 2 and C[2] = 4
        \item i = 2, place = C[0] = 1, so B[1] = 0 and C[0] = 0
        \item i = 1, place = C[6] = 10, so B[10] = 6 and C[6] = 9
        \item Finally, B = [0, 0, 1, 1, 2, 2, 3, 3, 4, 6, 6]
    \end{itemize}
    
\item \textbf{\textcolor{blue}{You are given a Max-Heap with $n$ entries. Assume all entries are distinct. Your goal is to find the {\em third largest} entry. One way would be to {\tt EXTRACT-MAX} twice and then {\tt MAXIMUM}. How long does this take? Find a better (by which we always mean faster for $n$ large) way.}}
    \\ As {\tt EXTRACT-MAX} takes $O(\lg(n))$ and {\tt MAXIMUM} takes $O(1)$ that method would take $2O(\lg(n))+O(1) = O(\lg(n)) time$. Another method: the $3^{rd}$ largest is in the depth 1 or 2, so in the position from 1 to 7 in the array. Simply sort these 7 elements in $O(1)$ time and take the third one.

\end{enumerate}
\end{document}