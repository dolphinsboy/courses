\documentclass{article}
\usepackage[utf8]{inputenc}
\usepackage{color}
\usepackage{hyperref}

\title{Homework 8}
\author{quentin mcgaw}
\date{November 2017}

\begin{document}

\maketitle

\begin{center}
    Project research: short summary of paper
\end{center}
\textbf{Title:} A Fistful of Bitcoins: Characterizing Payments Among Men with No Names
\\ \textbf{Abstract:} Bitcoin is a purely online virtual currency, unbacked by either physical commodities or sovereign obligation; instead, it relies on a combination of cryptographic protection and a peer-to-peer protocol for witnessing settlements. Consequently, Bitcoin has the unintuitive property that while the ownership of money is implicitly anonymous, its flow is globally visible. In this paper we explore this unique characteristic further, using heuristic clustering to group Bitcoin wallets based on evidence of shared authority, and then using re-identification attacks (i.e., empirical purchasing of goods and services) to classify the operators of those clusters. From this analysis, we characterize longitudinal changes in the Bitcoin market, the stresses these changes are placing on the system, and the challenges for those seeking to use Bitcoin for criminal or fraudulent purposes at scale.
\\ \textbf{Authors:}Sarah Meiklejohn, Marjori Pomarole, Grant Jordan, Kirill Levchenko, Damon McCoyy, Geoffrey M. and Voelker Stefan Savage
\\ \textbf{Link:} \url{https://cseweb.ucsd.edu/~smeiklejohn/files/imc13.pdf}
\\ \textbf{Summary:} This paper concerns the characterization of transactions on the Bitcoin network. Bitcoin addresses (or accounts if you prefer, although this is incorrect) are clustered into real entities depending on various parameters and several heuristics. Each category of services using Bitcoin, such as exchanges, vendors or gambling platforms, are then carefully analyzed in terms of transactions, total balance and influence. This also gives a great indication to all the services categories and major services that existed before 2013, which will help to know what to search for in the project. The heuristics described in this paper could also be used in our blockchain analysis to improve our results.

\end{document}