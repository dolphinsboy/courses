\documentclass[10pt]{article}
\pagestyle{empty}
\usepackage{minted}
\usepackage{fancyhdr}
\usepackage{lastpage}
\pagestyle{fancy}
\renewcommand{\headrulewidth}{0pt}
\cfoot[R]{\thepage~of~\pageref{LastPage}}
\addtolength{\oddsidemargin}{-.875in}
\addtolength{\evensidemargin}{-.875in}
\addtolength{\textwidth}{1.75in}
\addtolength{\topmargin}{-.875in}
\addtolength{\textheight}{1.75in}

\begin{document}

\begin{minted}[fontsize=\normalsize]{scala}
/* 1. Define a class A that implements the Ordered trait. A should define the compare method 
required by Ordered and should override the toString method so that it prints something 
sensible (identifying the object as a A object). Each object of the A class should be 
instantiated with an integer parameter, e.g. by saying new A(6). The result of comparing 
two A's should be based on the values of the integers they were created with. For example, the 
expression (new A(5)).compare(new A(6)) should return �1, because 5 is less than 6 (you can 
simply subtract the integer values that the objects were created with). This will allow two 
A objects to be compared using the usual < and > operators. (Note: Due to limitations of the 
type system, == will not be redefined to automatically work correctly. Use (x.compare(y) == 0) 
if you want to test equality between objects x and y). */
class A(a:Int) extends Ordered[A]{
  val a_string = a.toString
  var value:Int = a
  def compare(that:A):Int = {
    val diff = this.value - that.value
    if(diff < 0){
      return -1;
    }else if(diff > 0){
      return 1;
    }else{ //they are equal
      return 0;
    }
  }
  override def toString() =  "A<"+a_string+">"
}

/* 2. Define a class B that extends class A. Each object of the B class should be 
instantiated with two integer parameters, e.g. by saying new B(6,7). The value to use 
in the compare() method is the sum of the two integers that the object was created with. 
For example, (new B(4,5) < new A(7)) would return false, because 4+5 is not less than 7. 
You don't need to override compare(). You can, for example, create a value field that is 
used in the comparison (and do the same in class A). Be sure that (new A(7) < new B(4,5)) 
would return true, given your definition of A and B.*/
class B(a:Int, b:Int) extends A(a){
  val b_string = b.toString
  value = this.a + this.b
  override def toString() =  "B<"+a_string+","+b_string+">"  
}

/* 3. Define a generic class C[T] with the following properties:
- Instances of C[T] are covariantly subtyped. For example, given the above classes 
  A and B, anywhere a C[A] is expected a C[B] can be used.
- Each instance of C[T] should be created with a function of type Int=>T. For example, 
  if g is a function of type Int=>A, then the following declaration would work: 
  var c: C[A] = new C(g)
- C[T] has a method apply(), which takes an integer parameter x and returns the result 
  of calling f(x), where f was the function that the object was created with. For 
  example, given the above definition of  the c variable, the expression c.apply(3) 
  would return the result of calling g(3). */
class C[+T](f:Int=>T) {
  def apply(x:Int) = f(x)
}

/* 4. Define an abstract generic class Tree[T] for implementing binary search trees, such that:
- T must itself implement the Ordered trait, allowing two T's to be compared.
- Any class derived from the Tree[T] must implement an insert method that takes a 
  parameter of type T and returns a value of type Tree[T]. To do this, simply put 
  the following (and nothing else) within the Tree[T] class definition: 
  def insert(x:T):Tree[T] */
abstract class Tree[T <: Ordered[T]]{ // extends Ordered[Tree[T]]
  def insert(x:T):Tree[T]
}

/* 5. Define three case classes, Leaf[T], Node[T], and Empty[T] that extend Tree[T] such that:
  - Leaf[T] should be parameterized by a T object representing the label at the leaf.
  - Node[T] should be parameterized by a T object representing the label at the node and by 
    two Tree[T]'s representing the left and right subtrees.
  - Empty[T] should take no parameters. For example, the following creates a Tree[A] object:
    val MyTree = Node(new A(3), Leaf(new A(2)), Node(new A(5),Leaf(new A(4)),Empty()))
  - Each of the the above case classes should override the toString() method to have a 
    sensible printout. For Empty[T], toString() can simply return the empty string. toString() 
    for a Node[T] should be defined so that a tree is printed out in in�order fashion, which 
    means for an interior node, the left subtree should be printed, followed by the label at 
    the node, followed by the right subtree.
  - Each of the case classes should define an insert() method, 
    override def insert(x:T):Tree[T] =  ... that returns a new Tree[T] resulting from inserting 
    a label x into the current tree in the appropriate place, so that the result is still a 
    binary search tree. As you probably remember, a binary search tree is a binary tree with the 
    property that the label at an interior node is greater than the label of its left child (if 
    there is a left child) and not greater than the label of its right child (if there is one). 
    For example, calling Empty[T]'s insert(x) should return a leaf whose label is x. Calling 
    Leaf[T]'s insert(x) should return an interior node that has one leaf child and one empty 
    child (depending on the relative values of x and the current leaf's label). Don't worry 
    about keeping the search tree balanced in any way. */
case class Empty[T <: Ordered[T]]() extends Tree[T]{
  override def insert(x:T):Tree[T] = {
    return Leaf[T](x)
  }
  override def toString() = ""
}

case class Leaf[T <: Ordered[T]](label:T) extends Tree[T]{
  override def insert(x:T):Tree[T] = {
    if(x > label){
      return Node[T](label, Empty[T], Leaf[T](x))
    }else{ //x <= label
      return Node[T](label, Leaf[T](x), Empty[T])
    }
  }
  override def toString() = label.toString
}

case class Node[T <: Ordered[T]](label:T, left:Tree[T], right:Tree[T]) extends Tree[T]{
  override def insert(x:T):Tree[T] = {
    if(x > label){
      return Node(label, left, right.insert(x))
    }else{ //x <= label
      return Node(label, left.insert(x), right)
    }
  }
  override def toString() = "("+left.toString+","+label.toString+","+right.toString+")"
}


/* 6. In a singleton class named Part2, put the following.
  - A method applyTo10()that takes a parameter c of type C[A] and returns the result of 
    calling c.apply(10).
  - A generic method found(), parameterized by a type parameter T, such that T implements 
    the Ordered trait. found() should take a parameter y of type T and a parameter tr of 
    type Tree[T] and return true if y is found in tr and false otherwise. Since tr is a 
    binary search tree (above), searching for y is fast (linear in the depth of the tree). 
    Do not use == to check for equality on objects of type T, rather write (y.compare(x)==0) 
    to determine if y equals x.
  - A main() method that simply calls the test() method, below.
      - The following test() method (shown below)
      - The output should look like:
        A<3>
        B<4,5>
        A<10>
        B<11,12>
        ((A<2>,A<3>,),A<4>,B<4,1>)
        A<3>is found in ((A<2>,A<3>,),A<4>,B<4,1>)
        A<5>is found in ((A<2>,A<3>,),A<4>,B<4,1>)
        B<2,1>is found in ((A<2>,A<3>,),A<4>,B<4,1>)
        B<4,2>is not found in ((A<2>,A<3>,),A<4>,B<4,1>) */
object Part2{
  def applyTo10(c:C[A]) = c.apply(10)
  def found[T <: Ordered[T]](y:T, tr:Tree[T]):Boolean = {
    tr match {
      case Empty() => false
      case Leaf(label) => {
        if((label < y) || (label > y)){
          return false
        }else{ //label and y are equal
          return true
        }
      }
      case Node(label, left, right) => {
        if(y < label){
          return found(y, left)
        }else if(y > label){
          return found(y, right)
        }else{
          return true
        }
      }
    }
  }
  def test(){
    val c1 = new C((x:Int) => new A(x))
    println(c1.apply(3))
    
    val c2 = new C((x:Int) => new B(x+1, x+2))
    println(c2.apply(3))
    
    println(applyTo10(c1))
    println(applyTo10(c2)) //relies on covariant subtyping
    
    var t1:Tree[A] = Empty()
    t1 = t1.insert(new A(4))
    t1 = t1.insert(new A(3))
    t1 = t1.insert(new B(4,1))
    t1 = t1.insert(new A(2))
    println(t1)
    
    val a3 = new A(3)
    val a5 = new A(5)
    val b21 = new B(2,1)
    val b42 = new B(4,2)
    
    if (found(a3, t1))
      println(a3 + "is found in " + t1)
    else
      println(a3 + "is not found in " + t1)
           
    if (found(a5, t1))
      println(a5 + "is found in " + t1)
    else
      println(a5 + "is not found in " + t1)
      
    if (found(b21, t1))
      println(b21 + "is found in " + t1)
    else
      println(b21 + "is not found in " + t1)
      
    if (found(b42, t1))
      println(b42 + "is found in " + t1)
    else
      println(b42 + "is not found in " + t1)
  }
  def main(args:Array[String]):Unit = {
    test()
  }
}
\end{minted}

\end{document}