\documentclass[11pt]{article}
\pagestyle{empty}
\usepackage{color}
\usepackage{fancyhdr}
\usepackage{lastpage}
\usepackage{amsfonts}
\usepackage{amsmath}
\usepackage{enumitem}
\usepackage[options]{algorithm2e}
\pagestyle{fancy}
\renewcommand{\headrulewidth}{0pt}
\cfoot[R]{\thepage~of~\pageref{LastPage}}
\addtolength{\oddsidemargin}{-.875in}
\addtolength{\evensidemargin}{-.875in}
\addtolength{\textwidth}{1.75in}
\addtolength{\topmargin}{-.875in}
\addtolength{\textheight}{1.75in}

\begin{document}
\begin{center}
\textbf{Lambda calculus}
\end{center}
\\ $(\lambda \textcolor{red}{x}\ .\ \textcolor{red}{x}\ \textcolor{blue}{y})$ has \textcolor{red}{x bound} and \textcolor{blue}{y free}
\begin{itemize}
    \item x is \textcolor{blue}{FREE} in:
    \begin{itemize}
        \item $x$ if it is not bound in any other variable or constant
        \item $(E,F) \Leftrightarrow (x$ \textcolor{blue}{free} in E$)$ OR $(x$ \textcolor{blue}{free} in F$)$
        \item $(\lambda y\ .\ E) \Leftrightarrow 
            \left\{
                \begin{array}{ll}
                    x \mbox{ \textcolor{blue}{free} in E} \\
                    x \mbox{ and } y \mbox{ are different variables}
                \end{array}
            \right.$
    \end{itemize}
    \item x is \textcolor{red}{BOUND} in:
    \begin{itemize}
        \item $(E,F) \Leftrightarrow (x$ \textcolor{red}{bound} in E$)$ OR $(x$ \textcolor{red}{bound} in F$)$
        \item $(\lambda y\ .\ E) \Leftrightarrow (x$ \textcolor{red}{bound} in E$)$ \textbf{OR} 
            $\left\{
                \begin{array}{ll}
                    x \mbox{ \textcolor{blue}{free} in E} \\
                    x \mbox{ and } y \mbox{ are the same variable}
                \end{array}
            \right.$
    \end{itemize}
\end{itemize}
\\ \textbf{Substitutions:}
\begin{itemize}
    \item $x[M/x] = M$
    \item $y[M/x] = y$, where $y$ is a constant or variable other than $x$.
    \item $(E\ F)[M/x] = (E[M/x])(F[M/x])$
    \item $(\lambda x\ .\ E)[M/x] = \lambda x\ .\ E$
    \item $(\lambda y\ .\ E)[M/x] = 
            \left\{
                \begin{array}{ll}
                    \lambda y\ .\ (E[M/x]) if 
                                            \left\{
                                                \begin{array}{ll}
                                                    x \mbox{ not \textcolor{blue}{free} in E} \\
                                                    \ \ \ \ \ \ \mbox{OR} \\
                                                    y \mbox{ not \textcolor{blue}{free} in M}
                                                \end{array}
                                            \right. \\
                    \ \ \ \ \ \ \mbox{OR} \\
                    \lambda z\ .\ (E[z/y])[M/x] \mbox{ otherwise, where z is new and not \textcolor{blue}{free} in E or M}
                \end{array}
            \right.$
\end{itemize}
\\ \textbf{Conversions:}
\begin{itemize}
    \item \textbf{$\alpha$-conversion:} If $y$ not \textcolor{blue}{free} in E, $(\lambda x\ .\ E) \Leftrightarrow (\lambda y\ .\ E[y/x])$
    \item \textbf{$\beta$-conversion:} $(\lambda x\ .\ E)M \Leftrightarrow E[M/x])$
    \item \textbf{$\eta$-conversion:} If 
                                        $\left\{
                                            \begin{array}{ll}
                                                x \mbox{ not \textcolor{blue}{free} in E} \\
                                                E \mbox{ denotes a function}
                                            \end{array}
                                        \right.$, then $(\lambda x\ .\ Ex) \Leftrightarrow E$
\end{itemize}
\\ \textbf{Reductions:}
$\beta$-conversion and $\eta$-conversion are called \textbf{reductions} when used in normal order.
\\ In example, $(\lambda x\ .\ E)M$ can be $\beta$-reduced and is therefore called a reducible expression, or \textbf{redex}.
\\ When an expression is no longer a redex, it is in its \textbf{normal form}.
\\ Example of an expression which reduces to normal form under normal-order evaluation, but not under applicative-order evaluation:
\\ The expression $(\lambda x.y)((\lambda z.z)(\lambda z.z))$.
\begin{itemize}
    \item Using \textbf{normal-order evaluation}:
    \\ $(\lambda x.y)((\lambda z.z)(\lambda z.z))$
    $\rightarrow y$
    \item Using \textbf{applicative-order evaluation}:
    \\ $(\lambda x.y)((\lambda z.z)(\lambda z.z))
    \rightarrow (\lambda x.y)((\lambda z.z)(\lambda z.z)) \rightarrow ... $ (infinite) 
\end{itemize}
\\ \textbf{Y-combinator:}
\\ The Y combinator is represented by 
$Y = (\lambda h.(\lambda x . h(x\ x))(\lambda x.h(x\ x)))$
\\ Now we have $Y f = (\lambda h.(\lambda x . h(x\ x))(\lambda x.h(x\ x))) f$
\\ $ = ((\lambda x . f(x\ x))(\lambda x.f(x\ x)))$
\\ $ = ((\lambda y . f(y\ y))(\lambda x.f(x\ x)))$ (here we replace x by y for a better understanding)
\\ $ = f ((\lambda x . f(x\ x))(\lambda x.f(x\ x)))$
\\ $ = f (Y\ f)$.
\\ Example of the definition of a recursive function using the Y-combinator and show how it is recursive with reductions.
\\ Let f = $\lambda$f.$\lambda$n.if($<$ n 3) 1 (+ (f (- n 1) (f (- n 2))
\\ Then FIBONACCI = Y f = f (Y f)
\\ For example, FIBONACCI 4 = (Y f) 4
\\ = f (Y f) 4
\\ = if($<$ 4 3) 1 (+ (Y f (- 4 1)) (Y f (- 4 2)))
\\ = (+ (Y f 3) (Y f 2)) (used a $\beta$-reduction)
\\ = (+ (if($<$ 3 3) 1 (+ (Y f (- 3 1)) (Y f (- 3 2)))) (if($<$ 2 3) 1 (+ (Y f (- 2 1)) (Y f (- 2 2)))))
\\ = (+ (+ (Y f 2) (Y f 1)) 1) (used a $\beta$-reduction)
\\ = (+ (+ (if($<$ 2 3) 1 (+ (Y f (- 2 1)) (Y f (- 2 2)))) (if($<$ 1 3) 1 (+ (Y f (- 1 1)) (Y f (- 1 2))))) 1)
\\ = (+ (+ 1 1) 1) (used a $\beta$-reduction)
\\ = (+ 2 1)
\\ = 3
\\ \textbf{Church-Rosser theorems:}
\\ The theorem states that $(E_1 \Leftrightarrow E_2) \Rightarrow \exists E\ |\ (E_1 \Rightarrow E)$ and $(E_2 \Rightarrow E)$.
\\ In other words, if there are two sequences of reductions that can be applied to the same term, then there exists a term that is reachable from both results by applying sequences of additional reductions.



\end{document}
