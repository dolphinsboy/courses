\documentclass[20pt]{article}
\pagestyle{empty}
\usepackage{minted}
\usepackage{fancyhdr}
\usepackage{lastpage}
\pagestyle{fancy}
\renewcommand{\headrulewidth}{0pt}
\cfoot[R]{\thepage~of~\pageref{LastPage}}
\addtolength{\oddsidemargin}{-.875in}
\addtolength{\evensidemargin}{-.875in}
\addtolength{\textwidth}{1.75in}
\addtolength{\topmargin}{-.875in}
\addtolength{\textheight}{1.75in}

\begin{document}

\begin{minted}[fontsize=\normalsize]{scheme}
;;; BASIC TYPES
> 12 ; Numbers (integers and floating point)
> 13.456 
> 'hello ; Symbol. The only characteristic of a symbol is its name
hello
"hello world" ; Strings
;;; Primary AGGREGATE TYPE is the list (not homogeneous !)
> '(34 45.6 tom sally "hello")
(34 45.6 tom sally "hello")

'(one (2 3 (4 five 6) 7) eight) ; nested lists

;;; EXPRESSIONS: atomic expressions and combinations
; Atomic expressions (not composed of any other expressions)
    ; numeric literals, symbol literals, boolean literals (#t #f), 
    ; the empty list: (), variable names
; Combinations: all start and end with parentheses (...)

;;; VARIABLE DEFINITION
(define x 6)
> 'hello ; Interpreter treats hello as a symbol, not a variable
hello
> hello ; error as there is no variable hello
> '(define y 7) ; Interpreter creates a list of symbols and a number.
(define y 7)
> y ; Error undefined
> (define y 7) ; Defines y

;;; ARITHMETIC expressions always use the prefix notation.
(+ x y) ; gives 13

;;; FUNCTION DEFINITION (define (<name> <param1> <param2> ...) <body>)
(define (f z y) (+ z (* 2 y)))
;;; FUNCTION CALL (<name> <arg1> <arg2> ...)
(f 3 (+ x 5)) ; gives 25

;;; CONDITIONALS: if and cond
(if (= x y) 'yes (* x y)) ; gives 42
(cond ((= x y) 'first)
      ((< x y) 'second) ; gives second
      ((> x y) 'third)
      (else 'fourth))

;;; TOWERS OF HANOI
(define (hanoi N from to temp)
  (cond ((= N 1) (display "Move disk from ") (display from)
                 (display " to ") (display to) (newline))
        (else (hanoi (- N 1) from temp to)
              (display "Move disk from ") (display from)
              (display " to ") (display to) (newline)
              (hanoi (- N 1) temp to from))))
> (hanoi 1 'a 'b 'c)
Move disk from a to b
> (hanoi 4 'a 'b 'c)
Move disk from a to c
Move disk from a to b
Move disk from c to b
Move disk from a to c
Move disk from b to a
Move disk from b to c
Move disk from a to c
Move disk from a to b
Move disk from c to b
Move disk from c to a
Move disk from b to a
Move disk from c to b
Move disk from a to c
Move disk from a to b
Move disk from c to b

;;; LISTS
'(1 2 3 4) ; gives (1 2 3 4)
'(1 (+ 2 3) (* x y)) ; QUOTE -> No evaluation, gives (1 (+ 2 3) (* x y))
(list 1 (+ 2 3) (* x y)) ; list -> evaluation, gives (1 5 42)
(cons 1 '(2 3 4 5)) ; gives (1 2 3 4 5)
(define myList '(2 3 4 5))
(cons 1 myList) ; gives (1 2 3 4 5)
myList ; gives (2 3 4 5) because CONS is purely FUNCTIONAL !
(append '(1 2 3 4) '(5 6 7 8)) ; gives (1 2 3 4 5 6 7 8)
(cons '(1 2 3 4) '(5 6 7 8)) ; gives ((1 2 3 4) 5 6 7 8)
(car '(2 4 6 8)) ; gives 2 (does not modify the list)
(cdr '(2 4 6 8)) ; gives (4 6 8)
(car (cdr (cdr '(1 2 3 4))))  ; gives 3 (third element)
'() ; the empty list, gives ()
(cdr '(1)) ; gives ()
(null? myList) ; gives #t is the list is empty, otherwise #f (gives #f)

;;; SHORTCUTS FOR ACCESSING LISTS
(car (cdr '(1 2 3 4))) ; gives 2
(cadr '(1 2 3 4))  ; gives 2
(car (cdr (cdr '(1 2 3 4)))) ; gives 3
(caddr '(1 2 3 4 5)) ; gives 3
(cadar '((1 2 3 4) (5 6 7) (8 9 10)))  ; gives 2

;;; RETURN Nth ELEMENT OF L
(define (nth n L)
    (cond ((= n 1) (car L)) ; base case is n = 1, return the first element
          (else (nth (- n 1) (cdr L))))) 
          ; assumption is that nth will work for n-1, construct the result for n. 
          ; The nth element of L is the (n-1)th element of (cdr L).
(nth 5 '(2 4 6 8 10 12)) ; gives 10

;;; MYAPPEND
  ; Define (myappend L1 L2) where L1 and L2 are lists.
  ; Base case:  If L1 is empty, then the result is just L2.
  ; Assumption:  If L1 has N elements, assume myappend works when it's first
  ;              argument has N-1 elements.  Note that (cdr L1) has N-1 elements.
  ; Step:  Since (myappend (cdr L1) L2) is assumed to work, it returns a list of
  ;        all the desired elements, except the first element of L1. So, we
  ;        CONS the first element of L1, (car L1), onto the list returned by
  ;        (myappend (cdr L1) L2)
(define (myappend L1 L2)  
  (cond ((null? L1) L2)  ;;; base case:  L1 is empty, so the result is just L2
        (else (cons (car L1) (myappend (cdr L1) L2)))))   

;;; ADD AN ELEMENT AT THE END OF A LIST
  ; Base Case:  L is the empty list, so just return the list containing x, i.e. (list x)
  ; Assumption: Assume (atEnd x (cdr L)) works.
  ; Step: Since (atEnd x (cdr L)) is missing the first element of L, CONS the (car L)
  ;       onto the result of (atEnd x (cdr L))
(define (atEnd x L) 
  (cond ((null? L) (list x))  
        (else (cons (car L) (atEnd x (cdr L))))))

;;; REVERSE
(reverse '(1 2 3 4 5)) ; gives (5 4 3 2 1)
(reverse '(1 2 (3 4) 5)) ; gives (5 (3 4) 2 1)
(define (myReverse L)
  (cond ((null? L) '())
        (else (append (myReverse (cdr L)) (list (car L))))))
(define (myReverse2 from to) ; Linear time !
  (cond ((null? from) to)
        (else (myReverse2 (cdr from) (cons (car from) to)))))
(myReverse2 '(1 2 3 4 5) '()) ; gives (5 4 3 2 1)
(define (myReverse L)
  (myReverse2 L '()))

;;; LET allows the definition of variables with a nested scope.
  ; (let ((<var1> <exp1>) (<var2> <exp2>) ... (<varN> <expN>)) <body>)
  ; evaluates <body> in an environment in which var1 has the value of exp2,
  ; var2 has the value of exp2, etc.
(let ((x 12) (y 3)) (+ x y)) ; gives 15
; The scope of the variables introduced by a LET is only within the body 
; of the LET.
(define (f x)
  (let ((y (+ x 1)) (z (* x 2)))
    (+ y z)))
(f 4) ; gives 5 + 8 = 13
; a variable definition in a let can refer to outer variables
(define x_global 100)
(let ((x_global 10) (y (* x_global 2))) ; gives 210 but don't change x_global
    (+ x_global y))
(let ((x 3) (y (+ x 1))) ; can't use x in the definition of y !
  (+ x y))
; We can use nested LET's to use new variables to define new variables
(let ((x 3))
  (let ((y (+ x 1)))
    (+ x y))) ; gives 7
; Or use LET* which does the same as nested LET's
(let* ((x 3) (y (+ x 1)) (z (* y 2)))
  (+ x y z)) ; gives 15

;;; FUNCTIONS are "first class" objects (can be treated as values)
;;; They can be passed as parameters, returned as the results of function calls, 
;;; placed into lists, etc.

;;; HIGHER ORDER function (function operating over functions)
(define (foo f x)
  (f (* x 2)))
(define (g y) (+ y 15))
(foo g 10) ; gives 35

;;; BUILT-IN FUNCTIONS
(map g '(20 30 40 50)) ; gives (35 45 55 65)             
(map car '((1 2 3) (a b c) ("hello" "goodbye"))) ; gives (1 a "hello")

;;; MYMAP
  ; Recursive thinking for (map f L)
  ; Base Case:  L is empty, return '()
  ; Assumption: (map f (cdr L)) returns a list of the results of
  ;             applying f to every element of (cdr L).
  ; Step:  Put the result of (f (car L)) on the front of the
  ;        result of (map f (cdr L))
(define (myMap f L)
  (cond ((null? L) '())
        (else (cons (f (car L)) (myMap f (cdr L))))))

(define (increment x) (+ x 1))
(map increment '(1 2 3 4)) ; gives (2 3 4 5)

;;; LAMBDAS
  ; (lambda (arg1 ... argN) body) evaluates to a function that takes 
  ; arg1 ... argN as parameters and executes body. That function has no name.
(lambda (x y) (+ x y))  ; gives #<procedure>
(map (lambda (x) (* x 2)) '(1 2 3 4 5))  ; gives (2 4 6 8 10)
;; Note "(define (f x y) (+ x y))" <-> (define f (lambda (x y) (+ x y)))

(define (f x)
  (lambda (y) (+ x y)))
(f 3) ; gives #<procedure>
(define g (f 3)) ; setting g to the result of calling (f 3), so g is a function.
(g 5) ; gives 8

;;; FUNCTIONS in LIST
(list (lambda (x) (+ x 1)) (lambda (x) (+ x 2)) (lambda (x) (+ x 3)))
; gives (#<procedure> #<procedure> #<procedure>)
(define L (list (lambda (x) (+ x 1)) (lambda (x) (+ x 2)) (lambda (x) (+ x 3))))
((car L) 3) ; gives 4
((caddr L) 3) ; gives 6


;;; Recursive functions within an inner scope (LET wont work)
(letrec ((f (lambda (x) (if (= x 0) 1 (* x (f (- x 1)))))))
  (f 5)) ; gives 120
; LETREC can be used to define mutually recursive functions
(letrec ((f (lambda (x) (if (= x 0) 1 (* x (g (- x 1))))))
         (g (lambda (x) (if (= x 0) 1 (* x (f (- x 1)))))))
  (g 6)) ; gives 720

; Final note: Scheme is not purely functional and can use set! for example.
\end{minted}

\end{document}