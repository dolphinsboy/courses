\documentclass[10pt]{article}
\pagestyle{empty}
\usepackage{minted}
\usepackage{fancyhdr}
\usepackage{lastpage}
\usepackage{listings}
\pagestyle{fancy}
\renewcommand{\headrulewidth}{0pt}
\cfoot[R]{\thepage~of~\pageref{LastPage}}
\addtolength{\oddsidemargin}{-.875in}
\addtolength{\evensidemargin}{-.875in}
\addtolength{\textwidth}{1.75in}
\addtolength{\topmargin}{-.875in}
\addtolength{\textheight}{1.75in}

\begin{document}

\begin{minted}[fontsize=\normalsize]{scala}
/* Anonymous functions (Lambda): ((parameters) => body) */
object lambdaExample{
  def higher(f: (Int) => Int) = f(f(3))  
  def main(args: Array[String]) = println(higher((x) => x+1)) // 5
}
// ----------------------------------------------------------------------------
/* Singleton class
   They are instantiated by a single object rather than by static members as in Java.
   They are declared using the "object" keyword. */

object Counter{
    var count = 0
    def increment{count = count + 1}
    def get() = count
    def useCount(f: (Int) => Unit) = f(count)
}

class F{
    val x = 12
    var y: Int = 20
    def f(z:Int):Int = x+y+z
    // Example of a method whose name is a symbol (no special syntax for that)
    def +(other:F):Int = y + other.y
    def set(w:Int){
        Counter.increment
        y = w
    }
}

object Bar{
    def printArg(x:Int){println("Got the value " + x)}
    def main(args: Array[String]){
        val f1 = new F()
        val f2 = new F()
        f1 set 17;  // same as f1.set(17)        
        f2.set(25)
        println(f1 + f2)  //same as f1.+(f2), prints 42
        println("Counter is now " + Counter.get()) // 2
        Counter.useCount(printArg) // Got the value 2
        Counter.useCount((x:Int) => println(x*1000)) // 2000
    }
}
// ----------------------------------------------------------------------------
/* Inheritance 
   All class inherit from Object */

class A(x:Int, y:String){ //parameters are used as for a constructor.
    val value = x
    val name = y
    override def toString():String = value + name
}
class B(a:Int, b:String, c:Int) extends A(a,b){
    val age = c
    override def toString() = age + (value + name + name)
}

object Test{
    def f(a:A){println(a)}
    def main(args:Array[String]){
        val a = new A(1, "One")
        val b = new B(2, "Two", 36)
        println(a)       
        f(b); //Subtyping: pass a B to a method that expects an A
    }
}
// ----------------------------------------------------------------------------
class A
class B extends A

object Test{
    def fB(f:B=>String){
        val b:B = new B
        f(b)
    }
    def fA(f:A=>String){
        val a:A = new A
        f(a)
    }
    def gA(x:A) = x.toString()
    def gB(y:B) = y.toString()
    def main(args: Array[String]){
        fA(gA)
        /* fA(gB) */ /* Passing a function of type B->string to a function 
                        expecting an A->string gives a type error */
        fB(gA) /* Passing a function of type A->String to a function 
                  expecting a B->String works thanks to contravariance */
        fB(gB)        
    }
}
// ----------------------------------------------------------------------------
/* Contravariance */
class A(x:Integer){
    override def toString() = "A<"+x+">"
}
class B(x:Integer, y:Integer) extends A(x){
    override def toString() = "B<"+x+","+y+">"
}
class Contra[-T](){
    var v:List[Any] = List()
    def insert(x:T){v = x::v}
    override def toString() = {
        def elementsToString(l:List[Any]):String = l match {
            case List() => ""
            case (x::xs) => x.toString() + " " + elementsToString(xs)
        }
        "Contra[ " + elementsToString(v) + "]"
    }
}

object Test{
    def test(c:Contra[B], z:B){
        c.insert(z)
    }
    def main(args: Array[String]){
        val cA = new Contra[A]()
        val cB = new Contra[B]()
        
        test(cB, new B(3,4))
        println(cB)

        cA.insert(new A(5))
        test(cA, new B(3,4)) //uses contravariant subtyping in first argument
        println(cA)
    }
}
// ----------------------------------------------------------------------------
/* COVARIANT TYPE PARAMETERS for generic classes */
class A{
    val x = 26
    override def toString() = "A: "+x
}
class B extends A{
    val y = 99
    override def toString() = "B: "+x+", "+y
}
class MyList[+T](lis:List[T]){
    val l: List[T] = lis
    def get = l
    def cons[SUPERT >: T](x:SUPERT): MyList[SUPERT] = new MyList[SUPERT](x::lis)
    /* Since cons can't have a T in contravariant position, it introduces 
       SUPERT that ranges over the supertypes of T. 
       The resulting list is a MyList[SUPERT] because a SUPERT is inserted. */
    // def cons(x:T):MyList[T] = new MyList[T](x::lis)
    // Fails because it contains a T in contravariant position
    def hd = l match {
        case (x::xs) => x
        case List() => throw new Exception()
    }
    def tl = l match {
        case (x::xs) => xs
        case List() => throw new Exception()
    }
    override def toString() = l + ""
}

object variant{  
    def foo(l:MyList[A]){
        println(l)
    }
    def main(args: Array[String]) {
        val lA = new MyList(List(new A()))
        val lB = new MyList(List(new B()))
        val hdA:A = lA.hd
        val hdB:A = lB.hd // expects A but receives B
        foo(lA) // expects MyList[A], prints List(A: 26)
        foo(lB) // expects MyList[A] and receives MyList[B], prints List(B: 26, 99)
        println(hdA) // A: 26
        println(hdB) // B: 26, 99
    }
}
// ----------------------------------------------------------------------------
/* Case classes and match clauses for pattern matching.
   An object of a case class can be created without the "new" keyword. */
abstract class Tree
case class Node(v:Int, l:Tree, r:Tree) extends Tree
case class Leaf(v:Int) extends Tree
case class Empty() extends Tree
object CaseClassTest{
    val myTree = Node(3,Leaf(4),Node(5,Leaf(6),Empty())) // no new !
    def printTree(t:Tree){
        t match {
            case Node(v,l,r) => print("(") 
                                printTree(l) 
                                print(v)
                                printTree(r)
                                print(")")
            case Leaf(v) => print(v)
            case Empty() => ()
        }
    }
    def main(args:Array[String]){printTree(myTree)} // (43(65))
}
// ----------------------------------------------------------------------------
/* TYPE CONSTRAINTS on type variables in generics */

trait CanBeCompared[T]{
    def <(other:T): Boolean
    def <=(other:T) = <(other) || ==(other)
    def >(other:T) = !(<(other)) && !=(other)
    def >=(other: T) = !(<(other))
}

/* TREE generic class where:
   - The type T of the label associated with the node implements 
     the CanBeCompared[T] trait, so that two labels of type T can
     be compared to each other
   - The tree class itself implements CanBeCompared[Tree[T]], so that
     two trees of type Tree[T] can be compared. */
abstract class Tree[T <: CanBeCompared[T]] extends CanBeCompared[Tree[T]]
case class Node[T <: CanBeCompared[T]](v:T, l:Tree[T], r:Tree[T]) extends Tree[T]{
    def <(other: Tree[T]) = other match{
        case Node(ov,ol,or) => (v < ov) && (l < ol) && (r < or)
        case _ => false
    }
    override def toString() = "("+l+")"+ v+"("+r+")"
}
case class Leaf[T <: CanBeCompared[T]](v:T) extends Tree[T]{
    def <(other:Tree[T]) = other match{
        case Leaf(ov) => v < ov
        case _ => false
    }
    override def toString() = v + ""
}
case class Empty[T <: CanBeCompared[T]]() extends Tree[T]{
    def <(other:Tree[T]) = other match{
        case Empty() => true
        case _ => false
    }
    override def toString() = ""
}
class MyInt(x:Int) extends CanBeCompared[MyInt]{
    val value = x
    def get = x
    def ==(other:MyInt) = x == other.get
    def <(other:MyInt) = x < other.get
    override def toString(): String = x + ""
}

object Test{
    def compare[T <: CanBeCompared[T]](a:T, b:T){
        if (a < b){
            println("a < b")
        }else{
            println("a > b")
        }
    }
    def main(args:Array[String]){
        val a = new MyInt(16)
        val b = new MyInt(7)
        compare(a,b) // Print a > b
        val x:Tree[MyInt] = Node(new MyInt(6), 
                                 Node(new MyInt(4),Leaf(new MyInt(3)),Leaf(new MyInt(5))), 
                                 Node(new MyInt(8),Leaf(new MyInt(7)),Empty()))
        val y:Tree[MyInt] = Node(new MyInt(16),
                                 Node(new MyInt(14),Leaf(new MyInt(13)),Leaf(new MyInt(15))),
                                 Node(new MyInt(18),Leaf(new MyInt(17)),Empty()))
        compare(x,y) // Prints a < b so x < y
    }
}
// ----------------------------------------------------------------------------
/* Code typed in class */
(x:Int) => x + 1 //Lambda
((x:Int, y:Int) => x + y)(3,4) // gives 7

class A
class B extends A  // B is a subclass of A
def fa(g:A=>Int) = g(new A())
def fb(g:B=>Int) = g(new B())
def p(x:A) = 6 //p is of type A->Int
def q(x:B) = 6   // q is of type B->Int
fa(p) // passing an A->Int to a function expecting an A->Int
fa(q) // passing a B->Int to a function expecting an A->Int, ERROR !
fb(q) // passing a B->Int to a function expecting an B->Int
fb(p) // passing an A->Int to a function expecting a B->Int, works fine (contravariant subtyping)
def f(g:Int=>A) = g(5) // f expects a parameter of type Int->A
f((x:Int) => new B()) // passing an Int->B to a function expecting an Int->A, works fine (covariance)
def fac(x:Int): Int = x match { // Factorial
     | case 0 => 1
     | case n => n * fac(n-1)
     | }

class C[T]
new C[Int]
// By default, there is no subtyping among instances of a generic.
def f(x:C[A]) = 3
f(new C[B]) // passing a C[B] to a function expecting a C[A], ERROR !

class C[+T] // Covariantly subtyped generic class
val c: C[A] = new C[B]() // using a C[B] object where a C[A] is expected (covariance)
val d: C[B] = new C[A]() // trying to use a C[A] object where a C[B] is expected (contravariance)
/* The + seriously limits the types of methods you can write.  Essentially, in a
   covariantly subtyped generic, the type parameter can only appear as the return
   type to a method (not the input type). */

class D[+T] {       
     |   def f(x:T) = 0  // ERROR ! since f takes a T as an input
     | }
// error: covariant type T occurs in contravariant position in type T of value x

class D[+T](x:T){
     |   def f(y:Int):T = x  // this will work, since T is the return type.
     | }

class E[-T]   // E[] is contravariantly subtyped
val c: E[A] = new E[B]() // Using an E[B] where an E[A] is expected. ERROR ! (covariance).
val c: E[B] = new E[A]()  // Using an E[A] where an E[B] is expected. Works fine (contravariance)
/* The - is also seriously limiting. In a contravariantly subtyped generic 
   class, a method can have the type parameter as the input type, but not 
   as the return type. */
class E[-T](x:T){ // contravariantly subtyped generic class
     |   def f(y:Int):T = x // ERROR ! can't have the type parameter T as a return type
     | }
class E[-T] {  
     |   def f(x:T): Int = 3  // this method works, can have type parameter T as the input type.
     | }
// ----------------------------------------------------------------------------
\end{minted}

\end{document}